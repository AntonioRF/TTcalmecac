\documentclass[10pt]{book}
\usepackage{cdt/cdtBusiness}
\usepackage{eld}
\usepackage{subfigure}
\usepackage{cite}
\usepackage{verbatim}
\usepackage{longtable}


%%%%%%%%%%%%%%%%%%%%%%%%%%%%%%%%%%%%%%%%%%%%%%%%%%%%%%%%%%%%%%%%
% Datos del proyecto

\title{Instituto Politécnico Nacional \\\bigskip Escuela Superior de Cómputo}
\subtitle{TLAMATINIME: Timetabling Problem, Prototipo de Optimización de Horarios en la ESCOM \\ 2017-B092}



\documento{C1--DA}{\Large{Que para cumplir con la opción de titulación curricular en la carrera de: \\ Ingeniería en Sistemas Computacionales
}}{\RELEASE{1.0}}%{\DRAFT{\today}} 

% Descomentar y establecer la fecha cuando se desee congelar la fecha del documento.
%\fecha{12 de Abril de 2013}

%%%%%%%%%%%%%%%%%%%%%%%%%%%%%%%%%%%%%%%%%%%%%%%%%%%%%%%%%%%%%%%%
% Documentos relacionados con el documento actual

% TODO: Escriba los documentos en los que está basado este documento.
\docRefs{

%     \docItem{Catálogo de Escuelas}{}{Catálogo de Escuelas de la DGAIR (Dirección General de Acreditación, Incorporación y Revalidación) de la SEP (Secretaría de Educación Pública)}
%     \docItem{Plan de Proyecto}{1.0}{ \cdtLabel{planProyecto}{Plan de proyecto del Programa de Acreditación de Escuelas Ambientalmente Responsables}}
%     \docItem{PAEAR}{1.0}{ \cdtLabel{PAEAR}{Programa de Acreditación de Escuelas Ambientalmente Responsables}}
%     \docItem{MPOCCT}{1.0}{ \cdtLabel{cct}{Manual de Procedmientos para la Operación del Catálogo de Centros de Trabajo}}
%     \docItem{M-1TR}{1.0}{ \cdtLabel{M-3TR}{Minuta de la Primera Reunión de Toma de Requerimientos}}
%     \docItem{M-2TR}{1.0}{ \cdtLabel{M-2TR}{Minuta de la Segunda Reunión de Toma de Requerimientos}}
%     \docItem{M-3TR}{1.0}{ \cdtLabel{M-3TR}{Minuta de la Tercera Reunión de Toma de Requerimientos }}
}

%%%%%%%%%%%%%%%%%%%%%%%%%%%%%%%%%%%%%%%%%%%%%%%%%%%%%%%%%%%%%%%%
% Elementos contenidos en el documento

% TODO: Al finalizar el análisis resuma aquí todos los elementos del componente: RN, CU, IU, MSG.
\elemRefs{
	%Resumen
	%\elemItem{Resumen}{1.0}{}
	%Contexto de trabajo
%	\elemItem{Contexto de trabajo}{1.0}{}
		%Problemática
	\elemItem{Problemática}{1.0}{}
		%Trabajo previo
	\elemItem{Trabajo previo}{1.0}{}
		%Solución propuesta
	\elemItem{Solución propuesta}{1.0}{}
		%Objetivos
	\elemItem{Objetivos}{1.0}{}
		%Justificación
	\elemItem{Justificación}{1.0}{}

	%Glosario de términos
 %--  \elemItem{Glosario de términos}{1.0}{Descripción de los términos técnicos y de negocio utilizados}


	


	
}

%%%%%%%%%%%%%%%%%%%%%%%%%%%%%%%%%%%%%%%%%%%%%%%%%%%%%%%%%%%%%%%%
\begin{document}

%=========================================================
% Portada
%\ThisLRCornerWallPaper{1}{cdt/theme/agua.jpg}
\thispagestyle{empty}

\maketitle
  
%=========================================================
% Hoja de revisión
%\makeDocInfo
\vspace{0.5cm}
%\makeElemRefs
%\makeDocRefs
%\makeObservaciones[3cm]
\vspace{0.5cm}
%\makeFirmas

%=========================================================
% Indices del documento
%\frontmatter
%\LRCornerWallPaper{1}{cdt/theme/pleca.jpg}
\tableofcontents
\listoffigures
%\listoftables
%\mainmatter

% Para esconder la información del documentador se descomenta el \hideControlVersion
% \hideControlVersion

%=========================================================
\chapter{Introducción}\label{chp:introduccion}
\cfinput{ModeloNegocios/EntornoTT/introduccion}

%=========================================================
\chapter{Marco Teórico}\label{chp:marcoTeo}
\cfinput{ModeloNegocios/EntornoTT/marcoTeorico}

%=========================================================
\chapter{Bosquejo General}\label{chp:bosquejoGeneral}
\cfinput{ModeloNegocios/EntornoTT/bosquejoGeneral}

%=========================================================
\chapter{Glosario de términos}\label{chp:glosarioTerminos}
\cfinput{ModeloNegocios/glosario}

%=====================MODELO DE ESTADOS==============================================================
%\chapter{Modelo de estados}\label{chp:modeloEstados} 
%\cfinput{ModeloNegocios/estados}

%=====================MODELO DE NEGOCIO==============================================================
\chapter{Modelo de negocio}\label{chp:modeloNegocios} 
\cfinput{ModeloNegocios/modelo}
\cfinput{ModeloNegocios/reglas}


%=====================MODELO DE COMPORTAMIENTO=======================================================
\chapter{Modelo de comportamiento}\label{chp:modeloComportamiento}
\cfinput{ModeloComportamiento/comportamiento}
%\cfinput{ModeloComportamiento/InicioSesion/modelo-comportamiento}


%=====================Prototipo 1=======================================================
%=================================ACADEMIAS====================================
\chapter{Prototipo 1: Academias}\label{chp:prototipo1}
%	\section{Modelo de negocio}
	\cfinput{ModeloNegocios/Academias}
	
%	\section{Modelo de comportamiento}
	\cfinput{ModeloComportamiento/Academias/modelo-comportamiento}
	
%	\section{Interfaces del módulo}
	\cfinput{ModeloInteraccion/Academias/AsistenciaAlumnos}
	
%   \section{Pruebas.}	
	\cfinput{ModeloComportamiento/Academias/Pruebas}

	

%=====================Prototipo 2=======================================================
%=================================INFRAESTRUCTURA==============================
\chapter{Prototipo 2: Infraestructura}\label{chp:prototipo2}
%	\section{Modelo de negocio}
	\cfinput{ModeloNegocios/Infraestructura}
	
%	\section{Modelo de comportamiento}
	\cfinput{ModeloComportamiento/Infraestructura/modelo-comportamiento}
	
%	\section{Interfaces del módulo}
	%	\subsection{Edificios}
		\cfinput{ModeloInteraccion/Infraestructura/Edificios}
		
	%	\subsection{Espacios}
		\cfinput{ModeloInteraccion/Infraestructura/Espacios}
	
%   \section{Pruebas.}	
	\cfinput{ModeloComportamiento/Infraestructura/Pruebas}
%=====================Prototipo 3=======================================================
%=================================OFERTA EDUCATIVA=============================
\chapter{Prototipo 3: Oferta educativa}\label{chp:prototipo3}
%	\section{Modelo de negocio}
	\cfinput{ModeloNegocios/OfertaEducativa}

%	\section{Modelo de comportamiento}
	\cfinput{ModeloComportamiento/OfertaEducativa/modelo-comportamiento}
	
	%\section{Interfaces del módulo}
	%	\subsection{Planes de Estudio}
		\cfinput{ModeloInteraccion/OfertaEducativa/PlanesEstudio}
		
	%	\subsection{Unidades de Aprendizaje}
		\cfinput{ModeloInteraccion/OfertaEducativa/UnidadesAprendizaje}

%   \section{Pruebas.}	
	\cfinput{ModeloComportamiento/Infraestructura/Pruebas}
	
%=====================Prototipo 4=======================================================
%=================================OFERTA EDUCATIVA=============================
\chapter{Prototipo 4: Profesores}\label{chp:prototipo4}
%	\section{Modelo de negocio}
\cfinput{ModeloNegocios/Profesores}

%	\section{Modelo de comportamiento}
\cfinput{ModeloComportamiento/Profesores/modelo-comportamiento}

%\section{Interfaces del módulo}
%	\subsection{Profesores}
\cfinput{ModeloInteraccion/Profesores/Profesores}

%   \section{Pruebas.}	
\cfinput{ModeloComportamiento/Infraestructura/Pruebas}
%=====================MODELO DE INTERACCIÓN======================================
\chapter{Modelo de interacción con el usuario}\label{chp:modeloInteraccionUsuario}

	\section{Diseño de mensajes}
	\cfinput{ModeloInteraccion/mensajes}

%=========================================================
%\chapter{Módulo de Optimización}\label{chp:moduloOpt}
%\cfinput{ModeloNegocios/EntornoTT/bosquejoGeneral}

%=========================================================
%\chapter{Pruebas de Integración}\label{chp:pruebasIntegracion}
%\cfinput{ModeloNegocios/EntornoTT/bosquejoGeneral}

%=========================================================
%\chapter{Trabajo futuro}\label{chp:trabajoFuturo}
%\cfinput{ModeloNegocios/EntornoTT/bosquejoGeneral}

\chapter{Bibliografía}\label{chp:bibliografia}
\cfinput{ModeloNegocios/EntornoTT/bibliografia}

%=========================================================

% Bibliografía
%\bibliographystyle{plain}
%\bibliography{bibSEIPA}
%\addcontentsline{toc}{chapter}{Bibliografía}



%\clossing
\end{document}
