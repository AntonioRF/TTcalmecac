\label{sec:introduccion}

\subsection{Antecedentes}
	\textbf{Survey on University Timetabling Problem} \\
	
	El problema de organización de tiempo de un curso en una universidad se cataloga como NP Hard, lo que significa que el problema se vuelve más difícil a medida que se aumenta el número de instancias aumenta, y disminuye la posibilidad de que exista una solución óptima. \\
	
	Se tienen restricciones que siempre deben cumplirse a toda costa tales como: el horario y los salones, pero se tienen condiciones que se pueden violar aunque sea penalizado como las preferencias de los profesores. \\ 
	
	Dadas sus características el problema puede ser abordado de distintas maneras entre ellas el uso de algoritmos basados en la naturaleza, simulaciones, distribuciones de colores y algoritmos genéticos. \\
	
	Categorías de problemas para su solución:
		
		\begin{itemize}
			\item UTTP: Problema de Horarios de Universidades
			en cuyo caso las clases son asignadas de forma semanal y se tienen que asignar materias a las clases dentro de un horario.
			
			\item CTTP: Problema de Horarios por Curso
			Se refiere al problema de las clases impartidas de un curso/materia a la semana en que el problema incluye asignarles salones y profesores a las materias dentro de un horario.
			
			\item LTTP: Problema de horarios por clase
			Se refiere al problema de asignar una sola clase de un curso por día en la universidad sin que se translapen unas con otras.
			
			\item ETTP: Problema de horarios por examinación
			Contiene una gran cantidad de situaciones propias de los demás problemas. La capacidad de los salones y el horario de profesores y alumnos debe ser tomado en cuenta.
		\end{itemize}
		
		Se han propuesto varias posibles soluciones para el problema tales como:
		
		\begin{itemize}
			
			\item PROGRAMACIÓN LINEAL(1989,1996,1997): Implementación secuencial de las restricciones del conflicto
			
			\item SATISFACCION DE RESTRICCIONES(1998): Se resuelven las restricciones con retroalimentación y consistencia.
			
			\item SOLUCIÓN DE COLOREADO DE GRAFOS(1974): Basado en la teoría de grafos se reduce el problema a el vértices y lados. Los eventos en forma de vértices y los lados en forma de condiciones.
			
			\item SOLUCIÓN METAHEURÍSTICA: Proceso de generación iterativa que utiliza varios conceptos de inteligencia para explorar las opciones viables del espacio de búsqueda mientras usa estrategias de aprendizaje para llevar a cabo un ordenamiento y encontrar de forma eficiente las soluciones óptimas
			
			\item ALGORITMO GENÉTICO: Algoritmo genético usado en combinación con búsqueda secuencial local que usa una fase de construcción y una fase de mejora. Donde cada posible juego de soluciones es conocida como población y cada posible solución es conocida como cromosoma. Se lleva una mejora con el algotimo híbrido con el algoritmo genético y de búsqueda local.
		\end{itemize}


\subsection{Problema de negocio}
	En la Escuela Superior de Cómputo la generación de la estructura educativa requiere de aproximadamente 2 meses, durante este tiempo se pueden realizar cambios en la asignación realizada incluso momentos antes de que aparezcan en el SAES. El tiempo estimado para la asignación manual de horarios equivale a 10 profesores por cada dos días, este proceso lo llevan a cabo los jefes de departamento y presidentes de academia. \\
	
	Con base en lo anterior, la forma en la que se genera la estructura educativa requiere de tiempo excesivo debido a que la asignación se hace de manera manual. Esta solución se considera la mejor desde el punto de vista de los jefes de academia, y depende totalmente de la perspectiva de cada uno de ellos.

\subsection{Solución}
	A fin de abordar el problema presentado, se propone la utilización de técnicas heurísticas y meta-heurísticas las cuales pueden proporcionar una solución adecuada. Es importante recalcar que si bien, puede existir una solución única, el espacio de la búsqueda es exponencial, por lo tanto encontrar esta solución podría tardar mucho tiempo. \\
	
	Este problema de asignación de horarios está relacionado con la complejidad de algoritmos NP-hard. Para el caso de la ESCOM, se cuentan con las variables: espacios (salones-laboratorios), profesores, unidades de aprendizaje, grupos, horas laborables. Con este conjunto de variables se estudiarán los diferentes algoritmos relacionas a los problemas NP-hard para obtener la función que nos permitirá ofrecer una solución al problema.

\subsection{Objetivos}
	Desarrollar una herramienta que permita generar una o varias opciones de configuración de horarios, tomando en cuenta las restricciones derivadas del análisis del proceso de generación de horarios de la Escuela Superior de Cómputo.

\subsection{Justificación}
	Para comprender mejor el proceso que la estructura educativa conlleva, nos entrevistamos con el subdirector académico M. en C. Iván Giovanny Mosso García, quien nos explicó los diferentes aspectos aquí expuestos. \\
	
	Con base en la entrevista realizada, concluimos que la ESCOM no cuenta con una herramienta en software que ayude a la generación automatizada de la estructura educativa.  \\
	
	El tiempo requerido es excesivo, los involucrados ven reducido el tiempo que pueden invertir en otras actividades. Un problema aún mayor radica en que la propuesta de la estructura educativa debe ser aprobada por la DAE, después de lo cual puede ser
	que se presenten cambios y esto implica un incremento de tiempo y esfuerzo del personal de la ESCOM en dicho proceso, mismos que aún con toda la experiencia que tienen pueden llegar a cometer errores debido a la complejidad del proceso.  \\
	
	De acuerdo a lo anterior la importancia del proyecto radica, esencialmente, en la asignación de recursos y tiempo que representaría dentro del proceso de generación de horarios, ya que se plantea que el proceso concluya con opciones viables de horarios de acuerdo a las características que se tienen en la escuela debido a que no solo se tiene que generar una propuesta de los horarios y esto no implica únicamente agrupar clases en bloques sino que el problema escala hasta la asociación de grupos, profesores, materias, salones, laboratorios y horas mismos que en conjunto son denominados horarios.
	