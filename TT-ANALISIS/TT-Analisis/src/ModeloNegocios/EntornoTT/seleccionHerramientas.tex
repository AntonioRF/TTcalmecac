\label{sec:seleccionHerramientas}
	
	En este capítulo se describirán las herramientas seleccionadas para el desarrollo de este trabajo. 
	
	\section{Lenguaje de programación}
		\textbf{Phyton:} En este lenguaje de programación diseñado por Guido Van Rossum en 1991, se desarrollará el algoritmo genético. Una de las ventajas que nos proporciona phyton es que es multiplataforma, es decir, puede implementarse e interoperar en múltiples plataformas. La versión de phyton utilizada es 2.7. [1]
	
		\textbf{LaTeX:} El documento en donde se concentrará todo el análisis de este trabajo será LaTeX, desarrollado por Leslie Lamport en 1984. LaTeX es multiplataforma y está orientado a la creación de documentos escritos, de modo que, estos  presentan una alta calidad tipográfica. La versión de LaTeX utilizada es LaTeX2e News Issue 27. [2]
		
		\textbf{DJango:}
		
		\textbf{PostgreSQL:}
	
	\section{Otras herramientas}
		En esta sección se describirán otras herramientas utilizadas para el desarrollo de este trabajo.
		
		\textbf{Balsamiq Mockups:} Balsamiq, una aplicación que facilita y agiliza la creación de bocetos.
		Balsamiq podríamos decir que es una aplicación/servicio pues no sólo cuenta con una aplicación nativa para OS X (también Windows y Linux) sino también con una versión web, de modo que podemos trabajar desde cualquier lugar. Y su finalidad no es otra que ayudar al desarrollo de aplicaciones con una herramienta que facilita la creación de esquemas.
	
	
	\begin{itemize}
		\item UTTP: Problema de Horarios de Universidades
		en cuyo caso las clases son asignadas de forma semanal y se tienen que asignar materias a las clases dentro de un horario.
		
		\item CTTP: Problema de Horarios por Curso
		Se refiere al problema de las clases impartidas de un curso/materia a la semana en que el problema incluye asignarles salones y profesores a las materias dentro de un horario.
		
		\item LTTP: Problema de horarios por clase
		Se refiere al problema de asignar una sola clase de un curso por día en la universidad sin que se translapen unas con otras.
		
		\item ETTP: Problema de horarios por examinación
		Contiene una gran cantidad de situaciones propias de los demás problemas. La capacidad de los salones y el horario de profesores y alumnos debe ser tomado en cuenta.
	\end{itemize}
	
	Se han propuesto varias posibles soluciones para el problema tales como:
	
	\begin{itemize}
		
		\item PROGRAMACIÓN LINEAL(1989,1996,1997): Implementación secuencial de las restricciones del conflicto
		
		\item SATISFACCION DE RESTRICCIONES(1998): Se resuelven las restricciones con retroalimentación y consistencia.
		
		\item SOLUCIÓN DE COLOREADO DE GRAFOS(1974): Basado en la teoría de grafos se reduce el problema a el vértices y lados. Los eventos en forma de vértices y los lados en forma de condiciones.
		
		\item SOLUCIÓN METAHEURÍSTICA: Proceso de generación iterativa que utiliza varios conceptos de inteligencia para explorar las opciones viables del espacio de búsqueda mientras usa estrategias de aprendizaje para llevar a cabo un ordenamiento y encontrar de forma eficiente las soluciones óptimas
		
		\item ALGORITMO GENÉTICO: Algoritmo genético usado en combinación con búsqueda secuencial local que usa una fase de construcción y una fase de mejora. Donde cada posible juego de soluciones es conocida como población y cada posible solución es conocida como cromosoma. Se lleva una mejora con el algotimo híbrido con el algoritmo genético y de búsqueda local.
	\end{itemize}


[1] https://www.python.org/
[2] https://www.latex-project.org/