\label{sec:seleccionHerramientas}
	
	En este capítulo se describirán las herramientas seleccionadas para el desarrollo de este trabajo. 
	
	\section{Lenguaje de programación}
		\textbf{Phyton:} En este lenguaje de programación diseñado por Guido Van Rossum en 1991, se desarrollará el algoritmo genético. Una de las ventajas que nos proporciona phyton es que es multiplataforma, es decir, puede implementarse e interoperar en múltiples plataformas. La versión de phyton utilizada es 2.7. [1] 
	
		\textbf{LaTeX:} El documento en donde se concentrará todo el análisis de este trabajo será LaTeX, desarrollado por Leslie Lamport en 1984. LaTeX es multiplataforma y está orientado a la creación de documentos escritos, de modo que, estos  presentan una alta calidad tipográfica. La versión de LaTeX utilizada es LaTeX2e News Issue 27. [2]
		
		\textbf{DJango:}
		
		\textbf{PostgreSQL:}
	
	\section{Otras herramientas}
		En esta sección se describirán otras herramientas utilizadas para el desarrollo de este trabajo.
		
		\textbf{Balsamiq Mockups:} Es una aplicación que facilita y agiliza la creación de bocetos. Esto nos permite crear las interfaces que muestran las gestiones del sistema. Balsamiq cuenta con una aplicación nativa para OS X (también Windows y Linux) y una versión web. La versión de Balsamiq Mockups utilizada es 3.5.15. [3] [4]
		
		\textbf{StarUML:} Es una herramienta que permite modelar los estándares UML. StarUML nos permitirá diagramar los casos de uso y crear diagramas de clases del sistema. La versión de StarUML utilizada es 2.8.1 [5]


	
	


	[1] https://www.python.org/
	[2] https://www.latex-project.org/
	[3] https://en.wikipedia.org/wiki/Balsamiq 
	[4] https://www.applesfera.com/aplicaciones-os-x-1/balsamiq-mockup-una-muy-buen-herramienta-para-esbozar-tus-futuras-apps
	[5] https://prezi.com/lxqgnl0h5m/que-es-staruml/
	[6] https://www.djangoproject.com/
	[7] https://developer.mozilla.org/es/docs/HTML/HTML5
	[8] https://developer.mozilla.org/es/docs/Web/CSS/CSS3
	[9] http://gs.statcounter.com/browser-market-share/all/mexico
	[10] http://gs.statcounter.com/os-market-share/desktop/worldwide