\label{sec:bibliografia}

	[1] https://www.python.org/
	[2] https://www.latex-project.org/
	[3] https://en.wikipedia.org/wiki/Balsamiq 
	[4] https://www.applesfera.com/aplicaciones-os-x-1/balsamiq-mockup-una-muy-buen-herramienta-para-esbozar-tus-futuras-apps
	[5] https://prezi.com/lxqgnl0h5m/que-es-staruml/
	[6] https://www.djangoproject.com/
	[7] https://developer.mozilla.org/es/docs/HTML/HTML5
	[8] https://developer.mozilla.org/es/docs/Web/CSS/CSS3
	[9] http://gs.statcounter.com/browser-market-share/all/mexico
	[10] http://gs.statcounter.com/os-market-share/desktop/worldwide
	[11] https://www.javascript.com/
	[12]http://desarrollowebydesarrolloweb.blogspot.mx/2015/02/tabla-comparativa-de-los-lenguajes-de.html
	[13]http://noticias.universia.com.ar/consejos-profesionales/noticia/2016/02/22/1136443/conoce-cuales-lenguajes-programacion-populares.html
	\label{14}[14]C. A. C. Coello, “Introducción a la computación evolutiva,” Notas del curso. Departamento de Ingeniería Eléctrica, Sección de Computación, Instituto Politécnico Nacional, México, 2004.
	\label{15}[15]Macías Duarte Carlos Antonio, "Análisis comparativo del desempeño de Técnicas Evolutivas aplicadas a la predicción de distribución de robos", Instituto Politécnico Nacional, México, 2016.
	\label{16}[16]Gregorio Toscano Pulido, "Optimización Multiobjetivo Usando un Micro Algoritmo Genético", Universidad Veracruzana, México, 2001.
	\label{17}[17]L. Araujo and C. Cervignón, "Algoritmos evolutivos: un enfoque práctico", Alfaomega, 2009.
	\label{18}[18] J. Pandey and A. K. Sharma, "Survey on University timetabling problem," 2016 3rd International Conference on Computing for Sustainable Global Development (INDIACom), New Delhi, 2016, pp. 160-164.
	\label{19}[19] S. Ribić, R. Turčinhožić and A. Muratović-Ribić, "Modelling constraints in school timetabling using integer linear programming," 2015 XXV International Conference on Information, Communication and Automation Technologies (ICAT), Sarajevo, 2015, pp. 1-6.
	\label{20}[20]  Michael Sipser (2013). Introduction to the Theory of Computation 3rd. Cengage Learning.
	\label{21}[21] R. Raghavjee and N. Pillay, "A comparison of genetic algorithms and genetic programming in solving the school timetabling problem," 2012 Fourth World Congress on Nature and Biologically Inspired Computing (NaBIC), Mexico City, 2012, pp. 98-103.	
	
		