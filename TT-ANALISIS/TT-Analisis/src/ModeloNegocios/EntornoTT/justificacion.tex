\label{sec:justificacion}

Para comprender mejor el proceso que la estructura educativa conlleva, nos entrevistamos con el subdirector académico M. en C. Iván Giovanny Mosso García, quien nos explicó los diferentes aspectos aquí expuestos. \\

Con base en la entrevista realizada, concluimos que la ESCOM no cuenta con una herramienta en software que ayude a la generación automatizada de la estructura educativa.  //

El tiempo requerido es excesivo, los involucrados ven reducido el tiempo que pueden invertir en otras actividades. Un problema aún mayor radica en que la propuesta de la estructura educativa debe ser aprobada por la DAE, después de lo cual puede ser
que se presenten cambios y esto implica un incremento de tiempo y esfuerzo del personal de la ESCOM en dicho proceso, mismos que aún con toda la experiencia que tienen pueden llegar a cometer errores debido a la complejidad del proceso. //

De acuerdo a lo anterior la importancia del proyecto radica, esencialmente, en la asignación de recursos y tiempo que representaría dentro del proceso de generación de horarios, ya que se plantea que el proceso concluya con opciones viables de horarios de acuerdo a las características que se tienen en la escuela debido a que no solo se tiene que generar una propuesta de los horarios y esto no implica únicamente agrupar clases en bloques sino que el problema escala hasta la asociación de grupos, profesores, materias, salones, laboratorios y horas mismos que en conjunto son denominados horarios.