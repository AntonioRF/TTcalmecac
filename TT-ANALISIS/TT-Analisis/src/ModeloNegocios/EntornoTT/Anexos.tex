\label{sec:Anexos}

\section{Anexo 1. Información recabada}

Por medio de el documento mostrado en la imagen \ref{imgPeticion} se solicitó al subdirector académico el Maestro en Ciencias Iván Giovanny Mosso García la información de la estructura académica del semestre anterior y del semestre actual, así como los siguientes datos:

\begin{itemize}
	\item Profesores:
		\begin{itemize}
			\item Nombre
			\item Primer Apellido
			\item Segundo Apellido
			\item Horario
			\item Academia a la que pertenece
			\item RFC
			\item Nombre del cargo(en caso de tener alguno)
		\end{itemize}	
	\item Unidades de aprendizaje:
		\begin{itemize}
			\item Nombre
			\item Clave
			\item Academia a la que pertenece
			\item Tipo de unidad de aprendizaje(Teórica, Práctica, Teórica-Práctica)
		\end{itemize}
	\item Infraestructura:
		\begin{itemize}
			\item Número de salones utilizables
			\item Nombre de los salones
			\item Número de salón
		\end{itemize}
\end{itemize}

En respuesta el profesor Iván nos proporcionó dos archivos en formato excel. El primero contiene las siguientes columnas con información referente a la estructura educativa:
	\begin{itemize}
		\item Profesor
		\item Departamento
		\item Grupo
		\item Unidad de aprendizaje
		\item Academia
		\item Salón
		\item Laboratorio
		\item Lunes
		\item Martes
		\item Miércoles
		\item Jueves
		\item Viernes
	\end{itemize}
De este formato se cuenta con dos páginas, una para la estructura del semestre 2018/2 que cuenta con 459 filas y otra para la estructura del semestre 2019/1 que cuenta con 474 filas. Cada fila de este formato representa la combinación de la información de todas las columnas señaladas previamente.\\

De este archivo la única columna que no utilizamos es la información del departamento al que pertenece el profesor y la información de los laboratorios en que se imparte la unidad de aprendizaje debido a que esta gestión en particular fue dejada fuera del alcanze del presente trabajo.\\

El segundo archivo que se nos proporcionó contiene la información de 306 unidades de aprendizaje. Las columnas de este archivo son:

	\begin{itemize}
		\item Clave
		\item Nombre de la unidad de aprendizaje
		\item Abreviatura de la unidad de aprendizaje
		\item Nivel
		\item Academia a la que pertenece
		\item Horas teoría
		\item Horas prácticas
		\item Plan al que pertenece
		\item Carrera a la que pertenece
	\end{itemize}

De este archivo tenemos que las horas prácticas y teóricas debemos sumarlas para obtener las horas totales de una unidad de aprendizaje que es lo que realmente necesitamos. Sin embargo no nos compete la información de el plan y la carrera a la que pertenecen ya que sólo contemplamos el plan de estudios actual de la ESCOM para la carrera de Ingeniería en Sistemas Computacionales. 