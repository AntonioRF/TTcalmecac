\label{sec:introduccion}

\textbf{Survey on University Timetabling Problem} \ref{18} \\

El problema de organización de tiempo de un curso en una universidad se cataloga como NP Hard, lo que significa que el problema se vuelve más difícil a medida que se aumenta el número de instancias aumenta, y disminuye la posibilidad de que exista una solución óptima. \\

Se tienen restricciones que siempre deben cumplirse a toda costa tales como: el horario y los salones, pero se tienen condiciones que se pueden violar aunque sea penalizado como las preferencias de los profesores. \\ 

Dadas sus características el problema puede ser abordado de distintas maneras entre ellas el uso de algoritmos basados en la naturaleza, simulaciones, distribuciones de colores y algoritmos genéticos. \\

Categorías de problemas para su solución:
	
	\begin{itemize}
		\item UTTP: Problema de Horarios de Universidades
		en cuyo caso las clases son asignadas de forma semanal y se tienen que asignar materias a las clases dentro de un horario.
		
		\item CTTP: Problema de Horarios por Curso
		Se refiere al problema de las clases impartidas de un curso/materia a la semana en que el problema incluye asignarles salones y profesores a las materias dentro de un horario.
		
		\item LTTP: Problema de horarios por clase
		Se refiere al problema de asignar una sola clase de un curso por día en la universidad sin que se translapen unas con otras.
		
		\item ETTP: Problema de horarios por examinación
		Contiene una gran cantidad de situaciones propias de los demás problemas. La capacidad de los salones y el horario de profesores y alumnos debe ser tomado en cuenta.
	\end{itemize}
	
	Se han propuesto varias posibles soluciones para el problema tales como:
	
	\begin{itemize}
		
		\item PROGRAMACIÓN LINEAL(1989,1996,1997): Implementación secuencial de las restricciones del conflicto
		
		\item SATISFACCION DE RESTRICCIONES(1998): Se resuelven las restricciones con retroalimentación y consistencia.
		
		\item SOLUCIÓN DE COLOREADO DE GRAFOS(1974): Basado en la teoría de grafos se reduce el problema a el vértices y lados. Los eventos en forma de vértices y los lados en forma de condiciones.
		
		\item SOLUCIÓN METAHEURÍSTICA: Proceso de generación iterativa que utiliza varios conceptos de inteligencia para explorar las opciones viables del espacio de búsqueda mientras usa estrategias de aprendizaje para llevar a cabo un ordenamiento y encontrar de forma eficiente las soluciones óptimas
		
		\item ALGORITMO GENÉTICO: Algoritmo genético usado en combinación con búsqueda secuencial local que usa una fase de construcción y una fase de mejora. Donde cada posible juego de soluciones es conocida como población y cada posible solución es conocida como cromosoma. Se lleva una mejora con el algotimo híbrido con el algoritmo genético y de búsqueda local.
	\end{itemize}
