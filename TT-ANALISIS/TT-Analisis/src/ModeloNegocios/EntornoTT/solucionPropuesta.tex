\label{sec:introduccion}

A fin de abordar el problema presentado, se propone la utilización de cómputo evolutivo. Es importante recalcar que si bien, puede existir una solución única, el espacio de la búsqueda es exponencial, por lo tanto encontrar esta solución podría tardar mucho tiempo o incluso podría darse el caso que no se llegue a la mejor solución. \\

Este problema de asignación de horarios está relacionado con la complejidad de algoritmos NP-hard, esto es, el tiempo y los recursos computacionales necesarios para llegar a la solución óptima se elevan de manera exponencial de aceurdo al número de variables, esto por el tamaño del espacio de búsqueda. Para el caso de la ESCOM, se cuentan con las variables: espacios (salones-laboratorios), profesores, unidades de aprendizaje, grupos, horarios. Con este conjunto de variables se estudiarán los diferentes algoritmos relacionas a los problemas NP-hard para obtener la función que nos permitirá ofrecer una solución al problema.\\

Debido a la complejidad particular del caso ESCOM en que se tienen alrededor de 200 profesores que son los que se tienen en la nómina, número que puede aumentar o disminuir de acuerdo a las necesidades de la escuela cada semestre, 88 unidades de aprendizaje que son el total de las aprobadas dentro del plan de estudios aunque se pueden o no impartir durante dicho semestre, 14 posibles horarios debido a que en ESCOM ya que casi todas las unidades de aprendizaje duran una hora y media y se imparten 3 días a la semana por lo que se han configurado los horarios en 14 posibles combinaciones de tres sesiones a la semana. Se tienen también al rededor de 86 salones aunque no todos sean utilizables para impartir clase, finalmente se generan al rededor de 82 grupos al semestre, atacando este problema por fuerza bruta, el total de posibles opciones es aproximadamente 1,737,612,800 combinaciones lo cuál lo vuelve un espacio de búsqueda demasiado grande como para poder atacarlo de manera tradicional. \\

Como parte de la solución, la primer aproximación será dividir el espacio de búsqueda de acuerdo a las restricciones de ESCOM, en primer lugar los profesores solo imparten un número finito de unidades de aprendizaje al semestre, en segundo lugar los grupos se asocian directamente a un salón, en tercer lugar las unidades de aprendizaje y grupos se dividen por nivel y finalmente los horarios de clase así como los profesores pueden ser matutinos o vespertinos. De esta manera cada particular espacio de búsqueda se reduce a alrededor de 54,600 posibilidades, de esta forma aún es un espacio demasiado grande para métodos tradicionales pero disminuye.\\

Entre las posibles soluciones usuales para este tipo de problemas se propone el uso de algoritmos genéticos sin embargo para este caso en particular la probabilidad de que el algoritmo llegue a la solución óptima es muy baja, podría ser también que la solución final no sea viable, sin mencionar tanto el costo en tiempo como el costo computacional de esto.\\

Finalmente después de analizarlo, se decidió que la mejor aproximación es utilizar programación genética con el operador de cruce en permutación, debido a que de esta manera al evaluar la viabilidad de la solución mientras está siendo creada se asegura llegar a una solución viable, que quizás no sea la mejor, pero al crear soluciones viables aseguramos que se soluciona el problema, y la segunda parte del algoritmo se enfoca a evaluar la solución de forma que se pueda definir con base en los criterios definidos por el usuario cuál es la mejor solución de entre las propuestas.\\