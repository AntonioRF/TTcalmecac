 % El tipo de regla de negocio (tercer parámetro del entorno 'BusinessRule') se describe en la siguiente tabla:
%---------------------------------------------------------------------------------------------------------------,
% TIPOS			|		DEFINICION						|	EJEMPLO												|
%---------------|---------------------------------------|-------------------------------------------------------|
% Habilitador   | La sentencia habilita o restringe 	| * Se pueden recibir solicitudes del tipo A, B y C.	|
%				| hacer algo o  una funcionalidad.		| * Se permite hacer algo si se tiene el estado X.		|
%---------------|---------------------------------------|-------------------------------------------------------|
% Cronometrado	| Se permite de manera controlada 		| * Se permiten hasta dos solicitudes del tipo D		| 
%				| por un contador.						|   por persona.										|
% 				|										| * El acceso al sistema se permite si no se tiene 		|
%  				|										|   más de X número de intentos fallidos.				|
%---------------|---------------------------------------|-------------------------------------------------------|
% Ejecutivo		| Autorizado por un superior, un perfil | * Se permite registrar extemporaneamente si lo 		|
%				| particular debe autorizar.			|   autoriza X.											|
%---------------|---------------------------------------|-------------------------------------------------------|
% Derivación	| Son de cálculo e inferencia, 			| * Un alumno irregular es aquel que tiene las 			|
%				| es un cálculo o conclusión derivados 	|   siguientes cacteristicas: A, B, C. 					|
%				| de un conjunto de datos. Puede ser una| * El formato de un correo o CURP.						|
%				| fórmula que dice cómo calcular algo 	|														|
%				| o el formato de un dato.				|														|
%---------------|---------------------------------------|-------------------------------------------------------|
% Restricción	| Restringe una funcionalidad o relación| * Traslape de fechas o periodos empalmados.			|
% 				| entre dos o mas objetos.				|														|
%  				|										|														|
%---------------------------------------------------------------------------------------------------------------'

% No editar las reglas cuyo estatus es APROBADO.

\newpage
\section{Reglas de negocio}
\subsection{Reglas derivadas del sistema}
%------------------------------------------------------------------------------------------------------------------
%============================== RN-S1=================================
\begin{BusinessRule}{RN-S1}{Datos requeridos}
	{Restricción}
	{Controla la operación}
	\BRitem{Versión}{1.0}
	\BRitem{Autor}{José Antonio Ricardo Flores}
	\BRitem{Estatus}{Terminado}
	\BRitem{Descripción}{Los datos proporcionados al sistema que son marcados como requeridos con el caracter *, no se deben omitir.}
	
\end{BusinessRule}

%============================== RN-S2 =================================
\begin{BusinessRule}{RN-S2}{Unicidad de elementos}
	{Restricción}
	{Controla la operación}
	\BRitem{Versión}{1.0}
	\BRitem{Autor}{José Antonio Ricardo Flores}
	\BRitem{Estatus}{Edición}
	\BRitem{Descripción}{En casos específicos, los datos proporcionados al sistema no se pueden duplicar ni registrar más de una vez.}
	
\end{BusinessRule}


%============================== Reglas derivadas del negocio =================================
\subsection{Reglas derivadas del negocio}

%------------------------------------------------------------------------------------------------------------------
%============================== RN-N1 =================================
\begin{BusinessRule}{RN-N1}{Número de niveles permitido para un edificio}
	{Restricción}
	{Controla la operación}
	\BRitem{Versión}{0.1}
	\BRitem{Autor}{Brenda Gómez Caballero}
	\BRitem{Estatus}{Edición}
	\BRitem{Descripción}{Un edificio podrá registrarse si y sólo si tiene como máximo 4 niveles.}
		
\end{BusinessRule}

%============================== RN-N2 =================================
\begin{BusinessRule}{RN-N2}{Número de espacios permitido para un nivel de edificio}
	{Restricción}
	{Controla la operación}
	\BRitem{Versión}{0.1}
	\BRitem{Autor}{Brenda Gómez Caballero}
	\BRitem{Estatus}{Edición}
	\BRitem{Descripción}{Un nivel de un edificio sólo podrá tener como máximo 30 espacios.}
		
	\end{BusinessRule}

%============================== RN-N3 =================================
\begin{BusinessRule}{RN-N3}{Calculo de la clave de un espacio}
	{Derivación}
	{Calculo}
	\BRitem{Versión}{0.1}
	\BRitem{Autor}{Brenda Gómez Caballero}
	\BRitem{Estatus}{Edición}
	\BRitem{Descripción}{La clave que se le asigna a un espacio estará dada por:
	Número del edificio + Número del nivel + Número del espacio.}
	\BRitem{Ejemplo}{Un espacio tiene las siguientes características: Se en encuentra en el edificio 1, en el nivel 1 y el número de espacio es 1. \\
		Por lo tanto, la clave será: "1101".
	}
	
\end{BusinessRule}

%============================== RN-N4 =================================
\begin{BusinessRule}{RN-N4}{Jefe de academia}
	{Restricción}
	{Controla la operación}
	\BRitem{Versión}{0.1}
	\BRitem{Autor}{Brenda Gómez Caballero}
	\BRitem{Estatus}{Edición}
	\BRitem{Descripción}{Un jefe de academia sólo podrá ser un profesor con base.}

\end{BusinessRule}

%============================== RN-N5 =================================
\begin{BusinessRule}{RN-N5}{Capacidad de un espacio}
	{Restricción}
	{Controla la operación}
	\BRitem{Versión}{0.1}
	\BRitem{Autor}{Brenda Gómez Caballero}
	\BRitem{Estatus}{Edición}
	\BRitem{Descripción}{Para un espacio la capacidad máxima debe ser de 35 alumnos.}
	
\end{BusinessRule}

%============================== RN-N6 =================================
\begin{BusinessRule}{RN-N6}{Eliminación de un espacio}
	{Restricción}
	{Controla la operación}
	\BRitem{Versión}{0.1}
	\BRitem{Autor}{Brenda Gómez Caballero}
	\BRitem{Estatus}{Edición}
	\BRitem{Descripción}{Un espacio no podrá ser eliminado si este ya fue asociado a un grupo.}
	
\end{BusinessRule}

%============================== RN-N7 =================================
\begin{BusinessRule}{RN-N7}{Eliminación de un edificio}
	{Restricción}
	{Controla la operación}
	\BRitem{Versión}{0.1}
	\BRitem{Autor}{Brenda Gómez Caballero}
	\BRitem{Estatus}{Edición}
	\BRitem{Descripción}{Un edificio no podrá ser eliminado si ya se le ha registrado al menos un espacio en alguno de sus niveles.}
	
\end{BusinessRule}

%============================== RN-N8 =================================
\begin{BusinessRule}{RN-N8}{Modificación de un espacio}
	{Restricción}
	{Controla la operación}
	\BRitem{Versión}{0.1}
	\BRitem{Autor}{Brenda Gómez Caballero}
	\BRitem{Estatus}{Edición}
	\BRitem{Descripción}{Un espacio no podrá ser modificado si este ya fue asociado a un grupo.}
	
\end{BusinessRule}

%============================== RN-N9 =================================
\begin{BusinessRule}{RN-N9}{Modificación de un edificio}
	{Restricción}
	{Controla la operación}
	\BRitem{Versión}{0.1}
	\BRitem{Autor}{Brenda Gómez Caballero}
	\BRitem{Estatus}{Edición}
	\BRitem{Descripción}{Un edificio no podrá ser modificado si ya se le ha registrado al menos un espacio en alguno de sus niveles.}
	
\end{BusinessRule}

%============================== RN-N10 =================================
\begin{BusinessRule}{RN-N10}{Nombre del plan}
	{Derivación}
	{Controla la operación}
	\BRitem{Versión}{0.1}
	\BRitem{Autor}{Carlos Aníbal Larios Moguel}
	\BRitem{Estatus}{Edición}
	\BRitem{Descripción}{El nombre del plan de estudios se conforma por la palabra plan concatenada al año en que el mismo fue aprobado.}
\end{BusinessRule}

%============================== RN-N11 =================================
\begin{BusinessRule}{RN-N11}{Máquina de estados del plan de estudios}
	{Derivación}
	{Controla la operación}
	\BRitem{Versión}{0.1}
	\BRitem{Autor}{Carlos Aníbal Larios Moguel}
	\BRitem{Estatus}{Edición}
	\BRitem{Descripción}{El plan de estudios puede encontrarse en alguno de los siguientes estados:
		\begin{Citemize}
			\item Aprobación: Es el estado en que el plan está siendo modificado aún por lo que no se puede usar para los horarios.
			\item Aprobado: Es el estado en que el plan ha sido aceptado y podrá ser considerado para realizar los horarios de los períodos siguientes.
			\item Vigente: Es el estado en que el plan se encuentra siendo utilizado, este es el estado que utilizaremos para crear los horarios.
			\item Derogado: Es el estado en que el plan ha sido remplazado por uno otro y deja de ser utilizado para crear horarios.
		\end{Citemize}}
\end{BusinessRule}

%============================== RN-N12 =================================
\begin{BusinessRule}{RN-N12}{Clave de Unidad de Aprendizaje}
	{Derivación}
	{Controla la operación}
	\BRitem{Versión}{0.1}
	\BRitem{Autor}{Carlos Aníbal Larios Moguel}
	\BRitem{Estatus}{Edición}
	\BRitem{Descripción}{La clave de la unidad de aprendizaje se construye con la una letra que identifica el programa académico al que pertence, seguido de el número correspondiente al nivel del plan de estudios al que pertenece y finalmente concatenado a un consecutivo de las unidades del nivel.}

\end{BusinessRule}

%============================== RN-N13 =================================
\begin{BusinessRule}{RN-N13}{Tipo de unidad de aprendizaje}
	{Derivación}
	{Controla la operación}
	\BRitem{Versión}{0.1}
	\BRitem{Autor}{Carlos Aníbal Larios Moguel}
	\BRitem{Estatus}{Edición}
	\BRitem{Descripción}{La unidad de aprendizaje puede ser de tipo obligatoria u optativa.}

\end{BusinessRule}

%============================== RN-N14 =================================
\begin{BusinessRule}{RN-N14}{Tipo de formación}
	{Derivación}
	{Controla la operación}
	\BRitem{Versión}{0.1}
	\BRitem{Autor}{Carlos Aníbal Larios Moguel}
	\BRitem{Estatus}{Edición}
	\BRitem{Descripción}{De acuerdo a las regulaciones del instituto, las unidades de aprendizaje pueden tener uno de los siguientes tipos de formación:
		\begin{Citemize}
			\item Formación Institucional
			\item Formación Científica-Básica
			\item Formación Profesional
			\item Formación Terminal e Integración
		\end{Citemize}}

\end{BusinessRule}

%============================== RN-N15 =================================
\begin{BusinessRule}{RN-N15}{Tipo de enseñanza}
	{Derivación}
	{Controla la operación}
	\BRitem{Versión}{0.1}
	\BRitem{Autor}{Carlos Aníbal Larios Moguel}
	\BRitem{Estatus}{Edición}
	\BRitem{Descripción}{De acuerdo a las regulaciones del instituto, las unidades de aprendizaje pueden tener uno de los siguientes tipos de enseñanza:
		\begin{Citemize}
			\item Teórica
			\item Práctica
			\item Teórica-Práctica
		\end{Citemize}
		.}

\end{BusinessRule}

%============================== RN-N16 =================================
\begin{BusinessRule}{RN-N16}{Horas de unidad de aprendizaje}
	{Habilitadora}
	{Controla la operación}
	\BRitem{Versión}{0.1}
	\BRitem{Autor}{Carlos Aníbal Larios Moguel}
	\BRitem{Estatus}{Edición}
	\BRitem{Descripción}{Solo las unidades de aprendizaje pueden tener sus horas por semana divididas entre horas teóricas y prácticas, las unidades teóricas y las unidades prácticas tendrán toda su carga de horas asignadas a dicho tipo de enseñanza.}

\end{BusinessRule}

