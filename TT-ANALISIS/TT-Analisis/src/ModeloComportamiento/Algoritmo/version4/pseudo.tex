\subsection{Prototipo4 del alogritmo de optimización}

Tomando en cuenta los resultados del tercer prototipo, hicimos una nueva versión del algoritmo, en esta ocasión los cambios realizados al algoritmo fueron para que en primer lugar el algoritmo considere los distintos niveles y turnos en que se distribuyen los horarios dentro de la ESCOM y en segundo lugar la información sea recibida de la base de datos generada por el sistema y que de igual manera el o los horarios que se generan como resultado del algoritmo sean almacenados en la base de datos.\\

En este caso tenemos que la población inicial se genera nivel por nivel y turno por turno, después de eso se evalúa de la misma manera que en el tercer prototipo previamente explicado exceptuando la ponderación de cada una de las penalizaciones, para finalmente establecer una mutación que de igual manera sea llevada a cabo nivel por nivel y turno por turno. Debido a la complejidad del problema la cuál ya fue explicada, al aumentar el tamaño de busqueda respecto al considerado en los prototipos anteriores, el tiempo que tarda el algoritmo, así como el número de iteraciones necesarias aumentó considerablemente y, de acuerdo a la experimentación, el criterio de alcanzar una calificación perfecta o en este caso 0 no es válido debido a que la probabilidad de no alcanzarla es alta.\\

Como se explicó anteriormente los algoritmos evolutivos tienen una estructura similar por lo que tenemos el siguiente pseudocódigo para representar nuestro algoritmo. De acuerdo a lo previamente mencionado se agregaron los criterios de paro necesarios para asegurar una solución, considerando el número de iteraciones resultante de las pruebas del tercer prototipo y considerando la evaluación de los horarios actuales de la ESCOM.\\

\begin{algorithm}[H]
	\DontPrintSemicolon
	\SetAlgoLined
	%\KwResult{Write here the result }
	inicializar los arreglos globales de valores MP, GS y H\;
	inicializar el numero de iteraciones N\;
	inicializar el tamaño de la población inicial IND\;
	inicializar el número de resultados deseados DESEADOS\;
	inicializar poblacionaux\;
	poblacion = poblacionEvaluada(IND,MP,H,GS)\;
	poblacionaux = poblacion\;
	y = 1\;
	i = 1\;
	stop = 0\;
	stop2 = 0\;
	\For{$y < = N$}{
		\For{$ i <= IND$}{
			\If{poblacionaux[i][1] != 0}
				aux = mutacionGrupos(mutacionHorarios(poblacionaux[i][0]))\;
				poblacionaux[i][0] = aux\;
				poblacionaux[i][1] = evalua(aux)\;
				\If{poblacion[i][1] == 0}{
					incrementar $stop$ en $1$\;
				}
				\If{poblacion[i][1] <= 150 and y > 1500}{
					incrementar $stop2$ en $1$\;
				}
				incrementar $i$ en $1$\;
		}
		\If{$stop == DESEADOS$}{
			break;
		}
		\If{$stop2 == DESEADOS$}{
			break;
		}
		incrementar $y$ en $1$\; 
	}
	return poblacionaux\;
	\caption{principal(número iteraciones,tamañoo poblacion, resultados deseados)}
\end{algorithm}

.

\subsubsection{Funci\'on: generar Poblacion}

Esta función genera una población inicial, los cambios a esta función son los siguientes, iniciando por filtrar a las materias y los grupos por nivel para poder así acomodarlos juntos, evitando así tener grupos con materias de niveles distintos. Teniendo en cuenta el turno de los profesores y los grupos, se les asigna un horario que también corresponda a dicho turno.\\

\begin{algorithm}[H]
	\DontPrintSemicolon
	\SetAlgoLined
	%\KwResult{Write here the result }
	inicializar variable Tamanio con el n\'umero de individuos que debe tener la poblaci\'on\;
	inicializar matrices binarias MG, PH, GH\;
	inicializar variable poblacion y la variable individuo como arreglos\;
	K = 1\;
	pmi = [0,0] \;
	t = 0\;
	g = 0\;
	profesor = 0\;
	materia = 0\;
	condicion = 0\;
	\While{$k < = Tamanio$}{
			
			$pmi \leftarrow $ individuo del arreglo MP sin repeticion \;
			materia = pmi[0]\;
			profesor = pmi[1]\;
			\While{$condicion  != 1$ }{
				$t \leftarrow $ individuo del arreglo H\;
				$g \leftarrow $	individuo del arreglo GS\;
				\If{MG[materia][g] == 0}{
					\If{PH[profesor][t] == 0}{
						\If{GH[g][t] == 0}{
							PH[profesor][t] = 1\;	
							MG[materia][g] = 1\;
							GH[g][t] = 1\;
							condicion = 1\;
							individuo = [pmi,g,t]\;
							poblacion[k]= individuo\;
						}
					}
				}
				incrementar $k$ en $1$\;
			}
		return poblacion\;	
	}
	
	\caption{generarPoblacion(MP,H,GS)}
\end{algorithm}

\subsubsection{Funci\'on: evalua}

Para evaluar a un individuo tomamos en cuenta las siguientes restricciones con su respectiva penalizacion. La penalizacion es mayor cuando se viola una restriccion escencial(hard constraint) y menor cuando se viola una restriccion no escencial(soft constraint).

\begin{itemize}
	\item Un profesor no puede dar clase en dos grupos al mismo tiempo. Penalizacion = (tamaño del arreglo de población inicial * 2) + 1
	
	\item No se pueden impartir dos clases al mismo tiempo en un grupo. Penalizacion = (tamaño del arreglo de población inicial * 2) + 1
	
	\item No se debe impartir dos veces la misma materia en un grupo. Penalizacion = (tamaño del arreglo de población inicial * 2) + 1
	
	\item El horario de un grupo debe tener los menos huecos posibles. Penalizacion = 1
	
	\item El horario de un profesor debe tener los menos huecos posibles. Penalizacion = 1
	
	\item Se debe evitar que un profesor imparta dos materias distintas en el mismo grupo. Penalizacion = 1
	
	La calificación que devuelve esta función es una sumatoria de todas las penalizaciones que se dan por infringir alguna restricción.
	
\end{itemize}


\begin{algorithm}[H]
	\DontPrintSemicolon
	\SetAlgoLined
	\SetKwBlock{Begin}{While}{}
	\SetAlgoLined
	
	%\KwResult{Write here the result }
	inicializar variable Tamanio con el n\'umero de individuos de la poblaci\'on que recibe\;
	inicializar matrices binarias MGE, PHE, GHE,PGE\;
	inicializar variable individuo como arreglo\;
	calificacion = 0\;
	profesor = 0\;
	materia = 0\;
	grupo = 0\;
	horario = 0\;
	k = 0\;
	\SetAlgoVlined \While{$k < = Tamanio$}{
		$individuo \leftarrow $ individuo de la población de individuos \;
		materia = individuo[0][0]\;
		profesor = individuo[0][1]\;
		grupo = individuo[1]\;
		horario = individuo[2]\;
		incrementar $phe[profesor][hora]$ en  $1$\;
		incrementar $mge[materia][grupo]$ en  $1$\;
		incrementar $ghe[grupo][hora]$ en $1$\;
		incrementar $pge[profesor][grupo]$ en $1$\;
		$pena = (tamaniopoblacioninicial * 2) + 1$\;
		\If{PHE[profesor][horario] $>$ 1  }{incrementar $calificacion$ en $pena$}
		\If{GHE[grupo][hora] $>$ 1}{incrementar $calificacion$ en $pena$}
		\If{MGE[materia][grupo] $>$ 1}{incrementar $calificacion$ en $pena$}
		\If{PGE[profesor][grupo]}{incrementar $calificacion$ en $1$}
		\If{PHE[profesor][1]==1 AND PHE[profesor][2]==0 AND PHE[profesor][3]==0}{ \If{PHE[profesor][4] == 1 OR PHE[profesor][5] == 1 OR PHE[profesor][6] == 1 OR PHE[profesor][7] == 1}{incrementar $calificacion$ en $1$}}
		\If{PHE[profesor][2]==1 AND PHE[profesor][3]==0 AND PHE[profesor][4]==0}{ \If{PHE[profesor][5] == 1 OR PHE[profesor][6] == 1 OR PHE[profesor][7] == 1}{incrementar $calificacion$ en $1$}}
		\If{PHE[profesor][3]==1 AND PHE[profesor][4]==0 AND PHE[profesor][5]==0}{ \If{PHE[profesor][6] == 1 OR PHE[profesor][7] == 1}{incrementar $calificacion$ en $1$}}
		\If{PHE[grupo][4]==1 AND PHE[grupo][5]==0 AND PHE[grupo][6]==0}{ \If{PHE[grupo][7] == 1}{incrementar $calificacion$ en $1$}}
	}
	\caption{evalua(individuos)}
\end{algorithm}

\begin{algorithm}
	\setcounter{AlgoLine}{44}
	\SetKwBlock{While}{}{end}
	\SetAlgoVlined
	%HUECOS GRUPO
	\While{
		\If{GHE[grupo][1]==1 AND GHE[grupo][2]==0 AND GHE[grupo][3]==0}{ \If{GHE[grupo][4] == 1 OR GHE[grupo][5] == 1 OR GHE[grupo][6] == 1 OR GHE[grupo][7] == 1}{incrementar $calificacion$ en $2$}}
		\If{GHE[grupo][2]==1 AND GHE[grupo][3]==0 AND GHE[grupo][4]==0}{ \If{GHE[grupo][5] == 1 OR GHE[grupo][6] == 1 OR GHE[grupo][7] == 1}{incrementar $calificacion$ en $2$}}
		\If{GHE[grupo][3]==1 AND GHE[grupo][4]==0 AND GHE[grupo][5]==0}{ \If{GHE[grupo][6] == 1 OR GHE[grupo][7] == 1}{incrementar $calificacion$ en $2$}}
		\If{GHE[grupo][4]==1 AND GHE[grupo][5]==0 AND GHE[grupo][6]==0}{ \If{GHE[grupo][7] == 1}{incrementar $calificacion$ en $2$}}
	}
	return calificacion\;	
	\caption{evalua(individuos)}
\end{algorithm}

\subsubsection{Función: Población evaluada}

Para generar varios individuos utilizando las funciones definidas se creó esta función, la cuál utiliza la función generar Población así como la función evalua. De forma que recibe como parámetros el tamaño de la población deseada y los arreglos materia-profesor, horario y grupo-salón. Utilizando dicha información y las funciones mencionadas, devuelve un una matriz con un número de filas igual al número de individuos determinado y donde la primer columna es contiene una estructura educativa y su segunda columna contiene la evaluación correspondiente a dicho individuo. Dicha matriz es nuestra población inicial con sus respectivas evaluaciones de forma que sea más sencillo manejar los resultados.\\

\begin{algorithm}[H]
	\DontPrintSemicolon
	\SetAlgoLined
	%\KwResult{Write here the result }
	inicializar las vairables aux y pcalif como arreglos\;
	inicializar variables tt y eva
	i = 0\;
	\For{$i <= IND$}{
		tt = generarPoblacion(MP,H,GS)\;
		eva = evalua(tt)\;
		aux = [tt,eva]\;
		pcalif[i] = aux\;
	}
	return pcalif\;
	\caption{poblacionEvaluada(IND,MP,H,GS)}
\end{algorithm}


\subsubsection{Funciones: mutación}


Se definieron dos funciones de mutación, en ambos casos estamos considerando que dado el espacio limitado que tenemos para generar dichos cambios, aplicamos la mutación al mismo tiempo en dos cromosomas distintos de esta manera se busca mantener la viabilidad del resultado. La primera función de mutación  actúa sobre el gen GS de manera que el profesor sigue impartiendo una materia en el mismo horario pero en un grupo distinto, lo cual mitiga los casos en que un profesor imparte dos materias distintas en el mismo grupo o los casos en que una materia se imparte dos veces en el mismo grupo. Tomando en cuenta lo anterior, la mutación se lleva a cabo únicamente dentro del mismo turno y del mismo nivel.\\ 


\begin{algorithm}[H]
	\DontPrintSemicolon
	\SetAlgoLined
	%\KwResult{Write here the result }
	inicializar variable aux1 y aux2 como arreglos\;
	grupo1 = 0\;
	grupo2 = 0\;
	$aux1 \leftarrow $ individuo de la poblacion\;
	$aux2 \leftarrow $ individuo de la poblacion\;
	grupo1 = aux1[1]\;
	grupo2 = aux2[1]\;
	aux1[1] = grupo2\;
	aux2[1] = grupo1\;
	poblacion[aux1] = aux1\;
	poblacion[aux2] = aux2\;
	return poblacion \;
	\caption{mutacionGrupos(poblacion)}
\end{algorithm}

La segunda función actua sobre el gen H manteniendo los profesores y las materias en el mismo grupo pero modificando de esta manera el horario en que la imparten, con esto se busca mitigar los traslapes en grupos y profesores a demás de reducir los huecos en los horarios de los profesores.\\

\begin{algorithm}[H]
	\DontPrintSemicolon
	\SetAlgoLined
	%\KwResult{Write here the result }
	inicializar variable aux1 y aux2 como arreglos\;
	horario1 = 0\;
	horario2 = 0\;
	$aux1 \leftarrow $ individuo de la poblacion\;
	$aux2 \leftarrow $ individuo de la poblacion\;
	horario1 = aux1[2]\;
	horario2 = aux2[2]\;
	aux1[2] = horario2\;
	aux2[2] = horario1\;
	poblacion[aux1] = aux1\;
	poblacion[aux2] = aux2\;
	return poblacion \;
	\caption{mutacionHorarios(poblacion)}
	\end{algorithm}  

\subsubsection{Resultado}

Como resultado de este prototipo tenemos que las funciones crean un horario mejor que el actual de ESCOM para el espacio de prueba utilizado de acuerdo a nuestros criterios, a demás esta versión funciona en conjunto con el sistema web.\\

Como parte negativa tenemos que el tiempo, así como el número de iteraciones que toma al algoritmo para llegar a la calificación perfecta, en este caso 0, es demasiado grande o incluso puede darse el caso en que no llega a dicha calificación, por lo tanto se definió a demás del criterio de paro de la calificación perfecta, tenemos un número de iteraciones definido de manera experimental.\\