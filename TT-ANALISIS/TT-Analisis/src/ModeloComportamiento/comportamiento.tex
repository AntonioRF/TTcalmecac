
%=========================================================
\section{Módulos del sistema}

    El sistema se encuentra organizado por módulos con la finalidad de agrupar y administrar de mejor manera los requerimientos funcionales del sistema. Dividir el sistema en módulos permite visualizar e identificar rápidamente aquellos aspectos funcionales que pueden tratarse conjuntamente. \\
%
%    La figura \ref{fig:ModulosPAEAR} muestra los módulos propuestos de manera inicial para el \saear. Cada uno de estos módulos agrupan los casos de uso que poseen funcionalidad similar o que trabajan en conjunto para alcanzar un aspecto funcional del sistema. Cada uno de los módulos que se muestran en la figura se describen a continuación:

%    \begin{figure}[h!]
%	\begin{center}
	     % \fbox{\includegraphics[width=\textwidth]{images/modulos.jpg}}
%	\caption{Módulos del \saear.}
%	\label{fig:ModulosPAEAR}
%	\end{center}
%    \end{figure}

    \begin{itemize}
    \item {\bf Módulo Academias:} Agrupa los casos de uso que tienen que ver con la gestión de las academias existentes en la Escuela Superior de Cómputo.
	
	\item {\bf Módulo Infraestructura:} Agrupa los casos de uso que tienen que ver con la gestión de la infraestructura, la cual incluye la gestión de los edificios y espacios de la Escuela Superior de Cómputo.

	\item {\bf Módulo Oferta Educativa:} Agrupa los casos de uso que permiten la gestión de la oferta educativa de la Escuela Superior de Cómputo, incluyendo la gestión de los planes de estudio y las unidades de aprendizaje.
    
	\item {\bf Módulo Profesores:} Agrupa los casos de uso que permiten la gestión de los profesores que imparten las unidades de aprendizaje en la Escuela Superior de Cómputo.	

	\item {\bf Módulo Configuración General:} Agrupa los casos de uso que permiten crear un proyecto para generar horarios y las restricciones que tendrán estos. 
	
	\end{itemize}

	

%=========================================================
\section{Actores del Sistema} \label{sec:Comportamiento:ActoresSistema}

Los actores son los perfiles asociados a las diversas áreas y/u organizaciones que intervienen en el proceso. Se han identificado los actores de acuerdo a las actividades y responsabilidades dentro del TLAMATINIME: Timetabling Problem, Prototipo de Optimización de Horarios en la ESCOM, los cuales se muestran en la figura \ref{fig:perfiles} y se describen a continuación.

%    \begin{figure}[h!]
%      \begin{center}
%	  \includegraphics[width=0.6\textwidth]{images/actores/Actores.png}
%      \caption{Perfiles identificados.}
%      \label{fig:perfilesPAEAR}
%      \end{center}
%    \end{figure}

%SUBDIRECTOR ACADÉMICO
\cdtLabel{Actor:SA}{}
\begin{actor}{Subdirector Académico}{Es la persona encargada de la planificación de la Estructura Educativa en la Escuela Superior de Cómputo.}

	\item[Área:] Subdirección Académica.
	\item[Responsabilidades:] \hspace{1pt}
		\begin{itemize}
		    \item Registrar, modificar y eliminar las academias existentes en la Escuela Superior de Cómputo.
		    
		   	\item Registrar, modificar, eliminar y consultar la infraestructura para la Escuela Superior de Cómputo, esto incluye a edificios y espacios.
		   	
		   	\item Registrar, modificar, eliminar y consultar los planes de estudio vigentes en la Escuela Superior de Cómputo.
		   	
		   	\item Registrar, modificar, eliminar y consultar las unidades de aprendizaje ofertadas para el plan de estudio vigente. 
		   	
		   	\item Registrar, modificar, eliminar y consultar a los profesores que son los encargados de impartir las unidades de aprendizaje ofertadas en la Escuela Superior de Cómputo. 
		   	
			\item Registrar, modificar, eliminar y consultar los proyectos para generar la Estructura Educativa.
			
			\item Ejecutar el algoritmo encargado de hacer los horarios.
			
		 \end{itemize}
	\item[Perfil:] \hspace{1pt}
		\begin{itemize}
		    \item Persona que conoce el proceso de la Estructura Educativa.
	    \end{itemize}
	\item[Cantidad:] Uno por la ESCOM
\end{actor}


%CAPTURISTA
\cdtLabel{Actor:C}{}
\begin{actor}{Capturista}{Es la persona encargada de capturar la información necesaria en el sistema, para generar la Estructura Educativa de la Escuela Superior de Cómputo.}
	
	\item[Área:] No existente.
	\item[Responsabilidades:] \hspace{1pt}
	\begin{itemize}
		\item Registrar, modificar y eliminar las academias existentes en la Escuela Superior de Cómputo.
		
		\item Registrar, modificar, eliminar y consultar la infraestructura para la Escuela Superior de Cómputo, esto incluye a edificios y espacios.
		
		\item Registrar, modificar, eliminar y consultar los planes de estudio vigentes en la Escuela Superior de Cómputo.
		
		\item Registrar, modificar, eliminar y consultar las unidades de aprendizaje ofertadas para el plan de estudio vigente. 
		
		\item Registrar, modificar, eliminar y consultar a los profesores que son los encargados de impartir las unidades de aprendizaje ofertadas en la Escuela Superior de Cómputo. 
		
	\end{itemize}
	\item[Perfil:] \hspace{1pt}
	\begin{itemize}
		\item Persona que captura información en el sistema.
	\end{itemize}
	\item[Cantidad:] Tres por la ESCOM.
\end{actor}


%JEFE DE DEPARTAMENTO
\cdtLabel{Actor:JD}{}
\begin{actor}{Jefe de Departamento}{Es la persona encargada de gestionar parte del proyecto para generar la Estructura Educativa de la Escuela Superior de Cómputo.}
	
	\item[Área:] Varias.
	\item[Responsabilidades:] \hspace{1pt}
	\begin{itemize}
		\item Registrar, modificar y consultar los proyectos para generar la Estructura Educativa.
		
		\item Ejecutar el algoritmo encargado de hacer los horarios.
		
	\end{itemize}
	\item[Perfil:] \hspace{1pt}
	\begin{itemize}
		\item Persona que realiza parte de la creación del proyecto para generar la Estructura Educativa en el sistema.
	\end{itemize}
	\item[Cantidad:] El jefe de cada uno de los departamentos de la ESCOM.
\end{actor}


%ASISTENTE DE LA Secretaría DE ADMINISTRACIÓN
%\cdtLabel{Actor:ASA}{}
%\begin{actor}{Asistente de la Secretaría de Administración}{-----}
%	
%	\item[Área:] Escuela Libre de Derecho - Secretaría de Administración.
%	
%	\item[Responsabilidades:] \hspace{1pt}
%	\begin{itemize}
%		\item Registrar en sistema las fechas para generar el calendario escolar de la ELD.
%	\end{itemize}
%	\item[Perfil:] \hspace{1pt}
%	\begin{itemize}
%		\item Conocer el área de Secretaría de Administración
%	\end{itemize}
%	\item[Cantidad:] Uno por la ELD.
%\end{actor}

%Secretaría DE ADMINISTRACIÓN
%\cdtLabel{Actor:SA}{}
%\begin{actor}{Secretaría de Administración}{-----}
%	
%	\item[Área:] Escuela Libre de Derecho - Secretaría de Administración.
%	
%	\item[Responsabilidades:] \hspace{1pt}
%	
%	\begin{itemize}
%		\item Revisar el calendario escolar registrado en el sistema.
%		\item Llevar a la Junta Directiva de la ELD la propuesta de calendario escolar para su aprobación.
%	\end{itemize}
%	\item[Perfil:] \hspace{1pt}
%	\begin{itemize}
%		\item Conocer el área de Secretaría de Administración
%	\end{itemize}
%	\item[Cantidad:] Uno por la ELD:
%\end{actor}
%
%
%%ENTREVISTADOR
%\cdtLabel{Actor:Entrevistador}{}
%\begin{actor}{Entrevistador}{-----}
%	
%	\item[Área:] ELD o externos
%	
%	\item[Responsabilidades:] \hspace{1pt}
%	
%	\begin{itemize}
%		\item Entrevistar a los aspirantes a la Escuela Libre de Derecho que le sean asignados.
%	\end{itemize}
%	\item[Perfil:] \hspace{1pt}
%	\begin{itemize}
%		\item Personal de la Escuela Libre de Derecho o Egresado.
%	\end{itemize}
%	\item[Cantidad:] Los requeridos para la cantidad de entrevistas necesarias.
%\end{actor}
%
%
%%COORDINADOR DE PSICOLOGOS
%\cdtLabel{Actor:CP}{}
%\begin{actor}{Coordinador de Psicólogos}{-----}
%	
%	\item[Área:] Externo
%	
%	\item[Responsabilidades:] \hspace{1pt}
%	
%	\begin{itemize}
%		\item Registrar resultados del examen electrónico.
%		\item Validar las evaluaciones de los psicólogos.
%	\end{itemize}
%	\item[Perfil:] \hspace{1pt}
%	\begin{itemize}
%		\item Licenciado en psicología.
%	\end{itemize}
%	\item[Cantidad:] Uno.
%\end{actor}
%
%
%%PSICOLOGO
%\cdtLabel{Actor:Psicologo}{}
%\begin{actor}{Psicólogo}{-----}
%	
%	\item[Área:] Externos
%	
%	\item[Responsabilidades:] \hspace{1pt}
%	
%	\begin{itemize}
%		\item Entrevistar a los aspirantes a la Escuela Libre de Derecho que le sean asignados.
%		\item Evaluar los resultados de las pruebas psicométricas.
%	\end{itemize}
%	\item[Perfil:] \hspace{1pt}
%	\begin{itemize}
%		\item Licenciado en psicología.
%	\end{itemize}
%	\item[Cantidad:] Los requeridos para la cantidad de entrevistas y evaluaciones necesarias.
%\end{actor}
%
%
%%PROFESOR
%\cdtLabel{Actor:Profesor}{}
%\begin{actor}{Profesor}{-----}
%	
%	\item[Área:] ELD
%	
%	\item[Responsabilidades:] \hspace{1pt}
%	
%	\begin{itemize}
%		\item Impartir clases a los alumnos de las materias que le fueron asignadas.
%	\end{itemize}
%	\item[Perfil:] \hspace{1pt}
%	\begin{itemize}
%		\item Profesor de la Escuela Libre de Derecho o Egresado.
%	\end{itemize}
%	\item[Cantidad:] Los requeridos para impartir las materias.
%\end{actor}
%
\newpage 
%%====================================================================================Estructuración académica
%
%\section{Casos de Uso del módulo de Gestionar Diplomados}
%
%La figura \ref{fig:casosUso:gestionarDiplomados} muestra los casos de uso que integran la funcionalidad del módulo de gestionar diplomados, el cual conlleva el registro, edición, eliminación tanto de los diplomados, versiones, módulos, temas y subtemas, además se puede realizar la visualización de la versión como el contenido de la misma, así como la finalización y derogación de una versión.
%\begin{figure}[htpb!]
%	\begin{center}
%		\fbox{\includegraphics[scale=0.2]{ModeloComportamiento/EstructuracionAcademica/imagenes/DCUEA-GestionarDiplomados.jpg}}
%		\caption{Diagrama de casos de uso del módulo de Gestionar diplomados. \label{fig:casosUso:gestionarDiplomados}}
%	\end{center}
%\end{figure}
%
%
%\section{Casos de Uso del módulo de Aprobar Versiones}
%
%La figura \ref{fig:casosUso:aprobarDiplomados} muestra los casos de uso que integran la funcionalidad del módulo de aprobar versiones donde se realiza la visualización de los módulos como el contenido de los mismos, para finalmente aprobar la versión o solicitar cambios de la misma.
%\begin{figure}[htpb!]
%	\begin{center}
%		\fbox{\includegraphics[width=0.9\textwidth]{ModeloComportamiento/EstructuracionAcademica/imagenes/DCUEA-AprobarDiplomado.jpg}}
%		\caption{Diagrama de casos de uso del módulo de Aprobar versiones. \label{fig:casosUso:aprobarDiplomados}}
%	\end{center}
%\end{figure}
%
%\section{Casos de Uso del módulo de Gestionar Especialidades}
%
%La figura \ref{fig:casosUso:gestionarEspecialidades} muestra los casos de uso que integran la funcionalidad del módulo de gestionar especialidades, el cual contiene el registro, edición, eliminación tanto de las especialidades, planes de estudio, materias como de los temas y subtemas, además se puede realizar la visualización, finalización y derogación de los planes de estudio, asi como la visualización de su contenido.
%\begin{figure}[htpb!]
%	\begin{center}
%		\fbox{\includegraphics[scale=0.2]{ModeloComportamiento/EstructuracionAcademica/imagenes/DCUEA-GestionarEspecialidades.jpg}}
%		\caption{Diagrama de casos de uso del módulo de Gestionar especialidad. \label{fig:casosUso:gestionarEspecialidades}}
%	\end{center}
%\end{figure}
%
%
%\section{Casos de Uso del módulo de Aprobar Planes de Estudios de Especialidad}
%
%La figura \ref{fig:casosUso:aprobarEspecialidad} muestra los casos de uso que integran la funcionalidad del módulo de aprobar planes de estudio de especialidad donde se realiza la visualización del contenido de los mismos, para finalmente aprobar el plan de estudios o solicitar cambios en el mismo.
%\begin{figure}[htpb!]
%	\begin{center}
%		\fbox{\includegraphics[width=0.9\textwidth]{ModeloComportamiento/EstructuracionAcademica/imagenes/DCUEA-AprobarEspecialidad.jpg}}
%		\caption{Diagrama de casos de uso del módulo de Aprobar planes de estudio de especialidad. \label{fig:casosUso:aprobarEspecialidad}}
%	\end{center}
%\end{figure}
%
%
%\section{Casos de Uso del módulo de Gestionar Maestrías}
%
%La figura \ref{fig:casosUso:gestionarMaestrias} muestra los casos de uso que integran la funcionalidad del módulo de gestionar maestrías, el cual conlleva el registro, edición, eliminación tanto de las maestrías, planes de estudio, materias, temas y subtemas, además se puede realizar la visualización, finalización y derogación de los planes de estudio, así como la visualización de su contenido.
%\begin{figure}[htpb!]
%	\begin{center}
%		\fbox{\includegraphics[scale=0.2]{ModeloComportamiento/EstructuracionAcademica/imagenes/DCUEA-GestionarMaestrias.jpg}}
%		\caption{Diagrama de casos de uso del módulo de Gestionar maestrías. \label{fig:casosUso:gestionarMaestrias}}
%	\end{center}
%\end{figure}
%
%
%\section{Casos de Uso del módulo de Instituciones}
%
%La figura \ref{fig:casosUso:gestionarInstituciones} muestra los casos de uso que integran la funcionalidad del módulo de instituciones, el cual conlleva el registro, edición, y eliminación de las instituciones externas a la Escuela Libre de Derecho, así mismo se podrán registrar y eliminar las materias de una institución.
%\begin{figure}[htpb!]
%	\begin{center}
%		\fbox{\includegraphics[scale=0.55]{ModeloComportamiento/EstructuracionAcademica/imagenes/DCUEA-Instituciones.jpg}}
%		\caption{Diagrama de casos de uso del módulo de Instituciones. \label{fig:casosUso:gestionarInstituciones}}
%	\end{center}
%\end{figure}
%
%
%\section{Casos de Uso del módulo de Aprobar Planes de Estudios de Maestría}
%
%La figura \ref{fig:casosUso:aprobarMaestria} muestra los casos de uso que integran la funcionalidad del módulo de aprobar planes de estudio de maestría donde se realiza la visualización del contenido de los mismas, para finalmente aprobar el plan de estudios o solicitar cambios en el mismo.
%\begin{figure}[htpb!]
%	\begin{center}
%		\fbox{\includegraphics[width=0.9\textwidth]{ModeloComportamiento/EstructuracionAcademica/imagenes/DCUEA-AprobarMaestria.jpg}}
%		\caption{Diagrama de casos de uso del módulo de Aprobar planes de estudio de maestría. \label{fig:casosUso:aprobarMaestria}}
%	\end{center}
%\end{figure}
%
%
%\section{Casos de Uso del módulo de Equivalencias}
%
%La figura \ref{fig:casosUso:gestionarEquivalencias} muestra los casos de uso que integran la funcionalidad del módulo de equivalencias, el cual conlleva el registro, edición, y eliminación de las equivalencias entre materias de una misma maestría entre diferentes planes de estudios o entre diferentes maestrías.
%\begin{figure}[htpb!]
%	\begin{center}
%		\fbox{\includegraphics[scale=0.7]{ModeloComportamiento/EstructuracionAcademica/imagenes/DCUEA-Equivalencias.jpg}}
%		\caption{Diagrama de casos de uso del módulo de Equivalencias. \label{fig:casosUso:gestionarEquivalencias}}
%	\end{center}
%\end{figure}
%
%
%\section{Casos de Uso del módulo de Equivalencias de Movilidad}
%
%La figura \ref{fig:casosUso:gestionarEquivalenciasMovilidad} muestra los casos de uso que integran la funcionalidad del módulo de equivalencias de movilidad, el cual conlleva el registro, edición y eliminación de las equivalencias entre materias de una maestría y materias de una institución externa a la Escuela Libre de Derecho.
%\begin{figure}[htpb!]
%	\begin{center}
%		\fbox{\includegraphics[scale=0.5]{ModeloComportamiento/EstructuracionAcademica/imagenes/DCUEA-EquivalenciasMovilidad.jpg}}
%		\caption{Diagrama de casos de uso del módulo de Equivalencias de Movilidad. \label{fig:casosUso:gestionarEquivalenciasMovilidad}}
%	\end{center}
%\end{figure}
%
%%================================================================================Gestión académica
%\section{Casos de Uso del módulo de Gestionar Oferta de Diplomado}
%
%La figura \ref{fig:casosUso:gestionarOfertaDiplomado} muestra los casos de uso que integran la funcionalidad del módulo de Gestionar Oferta de Diplomado, que se refieren a la gestión de oferta de diplomado, la cual contiene el registro, modificación, eliminación y finalización de las ofertas que pertenecen a un diplomado, así mismo el módulo contiene la configuración de los grupos, el cual contiene el registro, edición, eliminación, configuración de horarios y profesores y asignación de horario de grupo a los temas del diplomado.
%
%\begin{figure}[htpb!]
%	\begin{center}
%		\fbox{\includegraphics[width=.9\textwidth]{ModeloComportamiento/GestionAcademica/imagenes/DCUGADiplomado.jpg}}
%		\caption{Diagrama de casos de uso para el módulo de Gestionar Oferta de Diplomado. \label{fig:casosUso:gestionarOfertaDiplomado}}
%	\end{center}
%\end{figure}
%
%
%
%
%%\section{Casos de Uso del módulo de Gestionar Salones}
%%
%%La figura \ref{fig:casosUso:gestionarsalon} muestra los casos de uso que integran la funcionalidad del módulo de Gestionar Salones, que se refieren a la gestión, registro, edición y eliminación de los salón que pertenecen al área de Posgrado de la Escuela Libre de Derecho.
%%
%%\begin{figure}[htpb!]
%%	\begin{center}
%% 		\fbox{\includegraphics[width=.9\textwidth]{ModeloComportamiento/GestionAcademica/imagenes/DCUGA_Salones.png}}
%%		\caption{Diagrama de casos de uso para el módulo de Gestionar Salones. \label{fig:casosUso:gestionarsalon}}
%%	\end{center}
%%\end{figure}
%
%\section{Casos de Uso del módulo de Gestionar Inscripciones}
%
%La figura \ref{fig:casosUso:inscripcion} muestra los casos de uso que integran la funcionalidad del módulo de Gestionar Inscripciones, que se refieren a la apertura, cierre o reactivación de grupos y a la gestión de los aspirantes a grupos de programas académicos.
%
%\begin{figure}[htpb!]
%	\begin{center}
%		\fbox{\includegraphics[width=.9\textwidth]{ModeloComportamiento/InscripcionReinscripcion/imagenes/DCUIRInscripcion.jpg}}
%		\caption{Diagrama de casos de uso para el módulo de Gestionar Inscripciones. \label{fig:casosUso:inscripcion}}
%	\end{center}
%\end{figure}
%
%\section{Casos de Uso del módulo de Asistencia de Alumnos}
%
%La figura \ref{fig:casosUso:AsistenciaAlumnos} muestra los casos de uso que integran la funcionalidad del módulo de Asistencia de Alumnos, mediante los cuales se podrá editar y visualizar la asistencia de los alumnos en cada materia del programa académico en el que están inscritos, además de tener la posibilidad de imprimir una constancia de dicha asistencia.
%
%\begin{figure}[htpb!]
%	\begin{center}
% 		\fbox{\includegraphics[width=.9\textwidth]{ModeloComportamiento/EjecucionDePeriodoEscolar/imagenes/CUAsistenciadeAlumnos.png}}
%		\caption{Diagrama de casos de uso para el módulo de Asistencia de Alumnos. \label{fig:casosUso:AsistenciaAlumnos}}
%	\end{center}
%\end{figure}
%
%%%====================================================================================
%%\section{Casos de Uso del módulo de Generación de Convocatoria de Ingreso}
%%
%%La figura \ref{fig:casosUso:generacionConvocatoria} muestra los casos de uso que integran la funcionalidad del módulo de Generación de Convocatoria de Ingreso, que se refieren al registro, modificación y visualización de las convocatorias de ingreso que se pueden generar en los ciclos.
%%
%%\begin{figure}[h!]
%%	\begin{center}
%%		\fbox{\includegraphics[width=.9\textwidth]{ModeloComportamiento/modulo-admision/images/DCU_1-2Publicacion_deConvocatoria.png}}
%%		\caption{Diagrama de casos de uso para el módulo de Generación de Convocatoria de Ingreso. \label{fig:casosUso:generacionConvocatoria}}
%%	\end{center}
%%\end{figure}
%%
%%%====================================================================================
%%
%%
%%
%%\section{Casos de Uso del módulo de Cuenta de Usuario}
%%
%%La figura \ref{fig:casosUso:cuentaUsuario} muestra los casos de uso que integran la funcionalidad del módulo de Cuenta de usuario, que se refieren al registro de la información de un usuario o recuperación de contraseña.
%%
%%\begin{figure}[h!]
%%	\begin{center}
%%		\fbox{\includegraphics[width=.9\textwidth]{ModeloComportamiento/modulo-admision/images/DCU1_3CuentaUsuario.png}}
%%		\caption{Diagrama de casos de uso para el módulo de Cuenta de Usuario. \label{fig:casosUso:cuentaUsuario}}
%%	\end{center}
%%\end{figure}
%%
%%%====================================================================================
%%\section{Casos de Uso del módulo de Registro de Aspirantes}
%%
%%La figura \ref{fig:casosUso:registroAspirantes} muestra los casos de uso que integran la funcionalidad del módulo de Registro de aspirantes, que se refieren al registro, modificación y visualización de la información de los aspirantes como datos personales, información escolar, domicilio y medios de contacto.
%%
%%\begin{figure}[h!]
%%	\begin{center}
%%		\fbox{\includegraphics[width=.9\textwidth]{ModeloComportamiento/modulo-admision/images/DCU1_4RegistroAspirantes.png}}
%%		\caption{Diagrama de casos de uso para el módulo de Registro de Aspirantes. \label{fig:casosUso:registroAspirantes}}
%%	\end{center}
%%\end{figure}
%%
%%%====================================================================================
%%\section{Casos de Uso del módulo de Entrevistas}
%%
%%La figura \ref{fig:casosUso:entrevista} muestra los casos de uso que integran la funcionalidad del módulo de Entrevistas, que se refieren al registro, modificación y visualización de la información de las entrevistas agendadas así como la información de los entrevistadores asociados a este proceso.
%%
%%\begin{figure}[h!]
%%	\begin{center}
%%		\fbox{\includegraphics[width=.9\textwidth]{ModeloComportamiento/modulo-admision/images/DCU1_8PlanificacionEntrevistas.png}}
%%		\caption{Diagrama de casos de uso para el módulo de Entrevistas. \label{fig:casosUso:entrevista}}
%%	\end{center}
%%\end{figure}
%%
%%%====================================================================================
%%\section{Casos de Uso del módulo de Selección de Aspirantes}
%%
%%La figura \ref{fig:casosUso:seleccionAspirantes} muestra los casos de uso que integran la funcionalidad del módulo de Selección de aspirantes, que se refieren al registro, modificación y visualización de los aspirantes que fueron seleccionados para ingresar a la Escuela Libre de Derecho.
%%
%%\begin{figure}[h!]
%%	\begin{center}
%%		\fbox{\includegraphics[width=.9\textwidth]{ModeloComportamiento/modulo-admision/images/DCU1_9SeleccionAspirantes.png}}
%%		\caption{Diagrama de casos de uso para el módulo de Selección de Aspirantes. \label{fig:casosUso:seleccionAspirantes}}
%%	\end{center}
%%\end{figure}
%%
%%%====================================================================================
%%%\section{Casos de Uso del módulo de Planes de Estudios}
%%%
%%%La figura \ref{fig:casosUso:PlanesdeEstudios} muestra los casos de uso que integran la funcionalidad del módulo de Planes de Estudios de la  Escuela Libre de Derecho.
%%%
%%%\begin{figure}[h!]
%%%	\begin{center}
%%%		\fbox{\includegraphics[width=.9\textwidth]{ModeloComportamiento/modulo-estructura-academica/images/Planes_de_Estudios.jpg}}
%%%		\caption{Diagrama de casos de uso para el módulo de Planes de Estudios. \label{fig:casosUso:PlanesdeEstudios}}
%%%	\end{center}
%%%\end{figure}
%%%
%%%%====================================================================================
%%%\section{Casos de Uso del módulo de Equivalencias}
%%%
%%%La figura \ref{fig:casosUso:Equivalencias} muestra los casos de uso que integran la funcionalidad del módulo de las Equivalencias de Planes de Estudios de la  Escuela Libre de Derecho.
%%%
%%%\begin{figure}[h!]
%%%	\begin{center}
%%%		\fbox{\includegraphics[width=.9\textwidth]{ModeloComportamiento/modulo-estructura-academica/images/Equivalencias.jpg}}
%%%		\caption{Diagrama de casos de uso para el módulo de Equivalencias. \label{fig:casosUso:Equivalencias}}
%%%	\end{center}
%%%\end{figure}
%%%
%%%%%====================================================================================
%%%%\section{Casos de Uso del módulo de Gestión Académica}
%%%%
%%%%La figura \ref{fig:casosUso:gestionAcademica} muestra los casos de uso que integran la funcionalidad del módulo de Gestión Académica, que se refieren a la gestión, registro, edición y eliminación de grupos, salones y profesores dentro de la  Escuela Libre de Derecho.
%%%%
%%%%\begin{figure}[h!]
%%%%	\begin{center}
%%%%		\fbox{\includegraphics[width=.9\textwidth]{ModeloComportamiento/modulo-gestion-academica/images/DCUAA7Gestion-Academica.png}}
%%%%		\caption{Diagrama de casos de uso para el módulo de Gestión Académica. \label{fig:casosUso:gestionAcademica}}
%%%%	\end{center}
%%%%\end{figure}
%%%
%%%%====================================================================================
%%%\section{Casos de Uso del módulo de Gestión de Grupos}
%%%
%%%La figura \ref{fig:casosUso:gestionGrupos} muestra los casos de uso que integran la funcionalidad del módulo de Gestión Académica, que se refieren a la gestión, registro, edición y eliminación de grupos dentro de la Escuela Libre de Derecho.
%%%
%%%\begin{figure}[h!]
%%%	\begin{center}
%%%		\fbox{\includegraphics[width=.9\textwidth]{ModeloComportamiento/modulo-gestion-academica/images/1.png}}
%%%		\caption{Diagrama de casos de uso para el módulo de Gestión de Grupos. \label{fig:casosUso:gestionGrupos}}
%%%	\end{center}
%%%\end{figure}
%%%
%%%%====================================================================================
%\section{Casos de Uso del módulo de Gestión de Salones}
%
%La figura \ref{fig:casosUso:gestionSalones} muestra los casos de uso que integran la funcionalidad del módulo de Gestionar Salones, que se refieren a la gestión, registro, edición y eliminación de salones dentro del area de Posgrado de Escuela Libre de Derecho.
%
%\begin{figure}[h!]
%	\begin{center}
%		\fbox{\includegraphics[width=.9\textwidth]{ModeloComportamiento/GestionAcademica/imagenes/DCUGASalones.jpg}}
%		\caption{Diagrama de casos de uso para el módulo de Gestión de Salones. \label{fig:casosUso:gestionSalones}}
%	\end{center}
%\end{figure}
%%%
%%%%====================================================================================
%%%\section{Casos de Uso del módulo de Gestión de Profesores}
%%%
%%%La figura \ref{fig:casosUso:gestionProfesores} muestra los casos de uso que integran la funcionalidad del módulo de Gestión Académica, que se refieren a la gestión, registro, edición y eliminación de los  profesores dentro de la Escuela Libre de Derecho.
%%%
%%%\begin{figure}[h!]
%%%	\begin{center}
%%%		\fbox{\includegraphics[width=.9\textwidth]{ModeloComportamiento/modulo-gestion-academica/images/3.png}}
%%%		\caption{Diagrama de casos de uso para el módulo de Gestión de Profesores. \label{fig:casosUso:gestionProfesores}}
%%%	\end{center}
%%%\end{figure}
%%%
%%%%====================================================================================
%%%\section{Casos de Uso del módulo de Gestión Ausencias}
%%%
%%%La figura \ref{fig:casosUso:gestionAusencias} muestra los casos de uso que integran la funcionalidad del módulo de Gestión Académica, que se refieren a la gestión, registro y edición de las ausencias de los profesores dentro de la Escuela Libre de Derecho.
%%%
%%%\begin{figure}[h!]
%%%	\begin{center}
%%%		\fbox{\includegraphics[width=.9\textwidth]{ModeloComportamiento/modulo-gestion-academica/images/4.png}}
%%%		\caption{Diagrama de casos de uso para el módulo de Gestión de Ausencias. \label{fig:casosUso:gestionAusencias}}
%%%	\end{center}
%%%\end{figure}
%%%
%%%%====================================================================================
%%%\section{Casos de Uso del módulo de Inscripción/Reinscripción}
%%%
%%%La figura \ref{fig:casosUso:inscripcionReinscripcion} muestra los casos de uso que integran la funcionalidad del módulo de Inscripción/Reinscripción, que se refieren a la Inscripción de alumnos a primer grado, así como la reinscripción de alumnos de segundo a quinto grado dentro de la Escuela Libre de Derecho.
%%%
%%%\begin{figure}[h!]
%%%	\begin{center}
%%%		\fbox{\includegraphics[width=.9\textwidth]{ModeloComportamiento/modulo-inscripcion-reinscripcion/images/CUInscripcionReinscripcion.png}}
%%%			\caption{Diagrama de casos de uso para el módulo de Inscripción/Reinscripción. \label{fig:casosUso:inscripcionReinscripcion}}
%%%	\end{center}
%%%\end{figure}
