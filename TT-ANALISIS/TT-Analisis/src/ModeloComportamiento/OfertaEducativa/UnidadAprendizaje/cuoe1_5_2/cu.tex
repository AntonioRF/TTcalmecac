
\begin{UseCase}{CUOE1.5.2}{Modificar unidad de aprendizaje}{	
		Permite al actor modificar la información de una unidad de aprendizaje que forma parte de un plan de estudio dado.
	}
	\UCccitem{Versión}{0.1}
	\UCccsection{Administración de Requerimientos}
	\UCccitem{Autor}{Brenda Gómez Caballero}
	\UCccitem{Evaluador}{}
	\UCccitem{Operación}{Registro}
	\UCccitem{Prioridad}{Alta}
	\UCccitem{Complejidad}{Baja}
	\UCccitem{Volatilidad}{Muy baja}
	\UCccitem{Madurez}{Alta}
	\UCccitem{Estatus}{Edición}
	\UCccitem{Fecha del último estatus}{01 de Abril de 2018}
	
	% Copie y pegue este bloque tantas veces como revisiones tenga el caso de uso.
	% Esta sección la debe llenar solo el Revisor
	%--------------------------------------------------------
	\UCccsection{Revisión Versión 0.1} % Anote la versión que se revisó.
	\UCccitem{Fecha}{}
	\UCccitem{Evaluador}{}
	\UCccitem{Resultado}{}
	\UCccitem{Observaciones}{
		\begin{Titemize}
			\Titem \DONE 
			\Titem \DONE 
			\Titem \DONE 
		\end{Titemize}   
	}
	%-------------------------------------------------------------------
	\UCsection{Atributos}
	\UCitem{Actor(es)}{ \begin{UClist}
			\UCli \cdtRef{Actor:SA}{Subdirector Académico}
			\UCli \cdtRef{Actor:C}{Capturista}
		\end{UClist} 
	}
	\UCitem{Propósito}{Mantener actualizada la información de las unidades de aprendizaje, que serán utilizadas como elementos del algoritmo genético.}
	\UCitem{Entradas}{
		\begin{UClist}
			\UCli  \cdtRef{unidadAprendizaje:clave}{Clave}: \ioEscribir.
			\UCli  \cdtRef{unidadAprendizaje:nombre}{Nombre}: \ioEscribir.
			\UCli  \cdtRef{tipoUA:nombreTipoUA}{Tipo de unidad de aprendizaje}: \ioSeleccionar.
			\UCli  \cdtRef{tipoFormacion:nombre}{Tipo de formación}: \ioSeleccionar.
			\UCli  \cdtRef{academia:nombre}{Academia}: \ioSeleccionar.
			\UCli  \cdtRef{unidadAprendizaje:nivel}{Nivel}: \ioSeleccionar.
			\UCli  \cdtRef{tipoEnsenanza:nombre}{Tipo de enseñanza}: \ioSeleccionar.
			\UCli  \cdtRef{unidadAprendizaje:horasTeoricas}{Total de hora teóricas por semana}: \ioEscribir.
			\UCli  \cdtRef{unidadAprendizaje:horasPracticas}{Total de hora prácticas por semana}: \ioEscribir.
		\end{UClist}
	}
	\UCitem{Salidas}{
		\begin{UClist}
			\UCli  \cdtRef{unidadAprendizaje:clave}{Clave}: \ioObtener.
			
			\UCli  \cdtRef{unidadAprendizaje:nombre}{Nombre}: \ioObtener.
			
			\UCli  \cdtRef{tipoUA:nombreTipoUA}{Tipo de unidad de aprendizaje}: \ioObtener.
			
			\UCli  \cdtRef{tipoFormacion:nombre}{Tipo de formación}: \ioObtener.
			
			\UCli  \cdtRef{academia:nombre}{Academia}: \ioObtener.
			
			\UCli  \cdtRef{unidadAprendizaje:nivel}{Nivel}: \ioObtener.
			
			\UCli  \cdtRef{tipoEnsenanza:nombre}{Tipo de enseñanza}: \ioObtener.
			
			\UCli  \cdtRef{unidadAprendizaje:horasTeoricas}{Total de hora teóricas por semana}: \ioObtener.
			
			\UCli  \cdtRef{unidadAprendizaje:horasPracticas}{Total de hora prácticas por semana}: \ioObtener.
			
			\UCli \cdtIdRef{MSG1}{Operación Exitosa}: Se muestra en la pantalla \cdtIdRef{IUOE1.5}{Gestionar Unidades de Aprendizaje} indicando que la modificación se realizó correctamente.
		\end{UClist}
	}
	\UCitem{Precondiciones}{
		\begin{UClist}
			\UCli Que exista información en el catálogo de \textbf{Academia}.
		\end{UClist}
	}	
	\UCitem{Postcondiciones}{
		\begin{UClist}
			\UCli Ninguna	
		\end{UClist}
	}
	%Reglas de negocio: Especifique las reglas de negocio que utiliza este caso de uso
	\UCitem{Reglas de negocio}{
		\begin{UClist}
			\UCli \cdtIdRef{RN-S1}{Datos requeridos}.
			\UCli \cdtIdRef{RN-S2}{Unicidad de elementos}.
		\end{UClist}
	}
	\UCitem{Errores}{
		\begin{UClist}	
			\UCli \cdtIdRef{MSG3}{Falta dato obligatorio}: Se muestra en la pantalla \cdtIdRef{IUOE1.5.2}{Modificar Unidad de Aprendizaje} indicando que faltan campos obligatorios por completar.
			
			\UCli \cdtIdRef{MSG6}{Elemento duplicado}: Se muestra en la pantalla \cdtIdRef{IUOE1.5.2}{Modificar Unidad de Aprendizaje} indicando que la clave de la unidad de aprendizaje ya existe.
			
			\UCli \cdtIdRef{MSG7}{No existe información necesaria en el sistema}: Se muestra en la pantalla \cdtIdRef{IUOE1.5}{Gestionar Unidades de Aprendizaje} indicando que no existe información en el catálogo 'Academia'.
		\end{UClist}
	}
	\UCitem{Tipo}{Terciario, extiende del caso de uso \cdtIdRef{CUOE1.5}{Gestionar unidades de aprendizaje}.}
\end{UseCase}

\begin{UCtrayectoria}
	\UCpaso [\UCactor] Solicita modificar una unidad de aprendizaje oprimiendo el botón \cdtButton{Modificar} de la pantalla \cdtIdRef{IUOE1.5}{Gestionar Unidades de Aprendizaje}. 
	
	\UCpaso [\UCsist] Verifica que exista información en el catálogo \textbf{Academia}. \refTray{A}
	
	\UCpaso [\UCsist] Obtiene la información de la unidad de aprendizaje previamente registrada.
	
	\UCpaso [\UCsist] Solicita modificar la información a través de la pantalla \cdtIdRef{IUOE1.5.2}{Modificar Unidad de Aprendizaje} con la información obtenida.
	
	\UCpaso [\UCactor] Actualiza la información solicitada. \label{CUOE1.5.2:CompletaInfo}
	
	\UCpaso [\UCactor] Solicita modificar la información oprimiendo el botón \cdtButton{Aceptar}. \refTray{B}
	
	\UCpaso [\UCsist] Verifica que se hayan completado los campos obligatorios con base en la regla de negocios \cdtIdRef{RN-S1}{Datos requeridos}. \refTray{C}
	
	\UCpaso [\UCsist] Verifica que la clave de la unidad de aprendizaje no se haya registrado previamente, con base en la regla de negocios \cdtIdRef{RN-S2}{Unicidad de elementos}. \refTray{D}
	
	\UCpaso [\UCsist] Actualiza la información de la unidad de aprendizaje.
	
	\UCpaso [\UCsist] Muestra el mensaje \cdtIdRef{MSG1}{Operación Exitosa} en la pantalla \cdtIdRef{IUOE1.5}{Gestionar Unidades de Aprendizaje}, indicando que la unidad de aprendizaje se modificó correctamente.	
	
\end{UCtrayectoria}

\begin{UCtrayectoriaA}[Fin del caso de uso]{A}{El catálogo 'Academia' no tiene información.}
	\UCpaso [\UCsist] Muestra el mensaje \cdtIdRef{MSG7}{No existe información necesaria en el sistema} en la pantalla \cdtIdRef{IUOE1.5}{Gestionar Unidades de Aprendizaje}. 
\end{UCtrayectoriaA}

\begin{UCtrayectoriaA}[Fin del caso de uso]{B}{El actor desea cancelar la operación.}
	\UCpaso [\UCactor] Solicita cancelar la operación oprimiendo el botón \cdtButton{Cancelar}.
		
	\UCpaso [\UCsist] Muestra la pantalla \cdtIdRef{IUOE1.5}{Gestionar Unidades de Aprendizaje}.  
\end{UCtrayectoriaA}

\begin{UCtrayectoriaA}{C}{No se completaron los campos obligatorios.}
	\UCpaso [\UCsist] Muestra el mensaje \cdtIdRef{MSG3}{Falta dato obligatorio} en la pantalla \cdtIdRef{IUOE1.5.2}{Modificar Unidad de Aprendizaje}. 
	
	\UCpaso Regresa al paso \ref{CUOE1.5.2:CompletaInfo} de la trayectoria principal.
\end{UCtrayectoriaA}

\begin{UCtrayectoriaA}{D}{La clave de la unidad de aprendizaje ya fue registrada.}
	\UCpaso [\UCsist] Muestra el mensaje \cdtIdRef{MSG6}{Elemento duplicado} en la pantalla \cdtIdRef{IUOE1.5.2}{Modificar Unidad de Aprendizaje}.
	
	\UCpaso Regresa al paso \ref{CUOE1.5.2:CompletaInfo} de la trayectoria principal.
\end{UCtrayectoriaA}


