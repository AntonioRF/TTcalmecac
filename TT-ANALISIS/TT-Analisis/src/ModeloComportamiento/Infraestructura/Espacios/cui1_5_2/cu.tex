
\begin{UseCase}{CUI1.5.2}{Modificar Espacio}{	
	Permite al actor modificar un nuevo espacio de un edificio de la ESCOM.
}
\UCccitem{Versión}{0.1}
\UCccsection{Administración de Requerimientos}
\UCccitem{Autor}{Carlos Aníbal Larios Moguel}
\UCccitem{Evaluador}{}
\UCccitem{Operación}{Modificación}
\UCccitem{Prioridad}{Alta}
\UCccitem{Complejidad}{Baja}
\UCccitem{Volatilidad}{Muy baja}
\UCccitem{Madurez}{Alta}
\UCccitem{Estatus}{Edición}
\UCccitem{Fecha del último estatus}{2/Abril/2018}

% Copie y pegue este bloque tantas veces como revisiones tenga el caso de uso.
% Esta sección la debe llenar solo el Revisor
%--------------------------------------------------------
\UCccsection{Revisión Versión 0.1} % Anote la versión que se revisó.
\UCccitem{Fecha}{}
\UCccitem{Evaluador}{}
\UCccitem{Resultado}{}
\UCccitem{Observaciones}{
	\begin{Titemize}
		\Titem \DONE 
		\Titem \DONE 
		\Titem \DONE 
	\end{Titemize}   
}
%-------------------------------------------------------------------
	\UCsection{Atributos}
	\UCitem{Actor(es)}{ \begin{UClist}
			\UCli \cdtRef{Actor:SA}{Subdirector Académico}
			\UCli \cdtRef{Actor:C}{Capturista}
		\end{UClist} 
	}
	\UCitem{Propósito}{Modificar un nuevo espacio para poder asignarlo un grupo de Unidades de Aprendizaje.}
	\UCitem{Entradas}{
		\begin{UClist}
		\UCli \cdtRef{espacio:nivel}{Nivel}: \ioEscribir.
		\UCli \cdtRef{espacio:numero}{Número}: \ioEscribir.
		\UCli \cdtRef{espacio:nombre}{Nombre}: \ioEscribir.
		\UCli \cdtRef{espacio:capacidad}{Capacidad}: \ioEscribir.
		
		\UCli \cdtRef{tipoEspacio:nombre}{Tipo de espacio}: \ioSeleccionar.
		\UCli \cdtRef{tipoLaboratorio:nombreTipoLab}{Tipo de laboratorio}: \ioSeleccionar.
		\UCli \cdtRef{espacio:accesoDiscapacitados}{Acceso a discapacitados}: \ioSeleccionar.
		\UCli \cdtRef{espacio:funciona}{¿Puede ser utilizado?}: \ioSeleccionar.
		
		\UCli \cdtRef{espacio:observaciones}{Observaciones}: \ioEscribir.
		\end{UClist}
	}
	\UCitem{Salidas}{
		\begin{UClist}
			\UCli \cdtRef{espacio:nivel}{Nivel}.
			\UCli \cdtRef{espacio:numero}{Número}.
			\UCli \cdtRef{espacio:nombre}{Nombre}.
			\UCli \cdtRef{espacio:capacidad}{Capacidad}.
			
			\UCli \cdtRef{tipoEspacio:nombre}{Tipo de espacio}.
			\UCli \cdtRef{tipoLaboratorio:nombreTipoLab}{Tipo de laboratorio}.
			\UCli \cdtRef{espacio:accesoDiscapacitados}{Acceso a discapacitados}.
			\UCli \cdtRef{espacio:funciona}{¿Puede ser utilizado?}.
			
			\UCli \cdtRef{espacio:observaciones}{Observaciones}.
			\UCli \cdtIdRef{MSG1}{Operación Exitosa}: Se muestra en la pantalla \cdtIdRef{IUI1.5}{Gestionar Espacios} indicando que el registro se realizó correctamente.
		\end{UClist}
	}
	\UCitem{Precondiciones}{
		\begin{UClist}
			\UCli Que se haya registrado al menos un edificio.
		\end{UClist}
	}
	
	\UCitem{Postcondiciones}{
		\begin{UClist}
			\UCli Ninguna	
		\end{UClist}
	}
	
	%Reglas de negocio: Especifique las reglas de negocio que utiliza este caso de uso
	\UCitem{Reglas de negocio}{
		\begin{UClist}
			\UCli \cdtIdRef{RN-S1}{Datos requeridos}
			\UCli \cdtIdRef{RN-S2}{Unicidad de elementos} 
		\end{UClist}
	}
	\UCitem{Errores}{
		\begin{UClist}	
			\UCli \cdtIdRef{MSG3}{Falta dato obligatorio}: Se muestra en la pantalla \cdtIdRef{IUI1.5.2}{Modificar Espacio} indicando que faltan campos obligatorios por completar.
			
			\UCli \cdtIdRef{MSG6}{Elemento duplicado}: Se muestra en la pantalla \cdtIdRef{IUI1.5.2}{Modificar Espacio} indicando que el nombre de espacio ya existe.
			
			
			\UCli \cdtIdRef{MSG7}{No existe información necesaria en el sistema}: Se muestra en la pantalla \cdtIdRef{IUI1.5}{Gestionar Espacios} indicando que no se puede modificar un espacio debido a que no se ha registrado al menos un tipo de laboratorio.
		\end{UClist}
	}
	\UCitem{Tipo}{Primario}
\end{UseCase}

\begin{UCtrayectoria}
	\UCpaso [\UCactor] Solicita modificar un espacio dando clic en el ícono  IUmodificar 
	de la pantalla \cdtIdRef{IUI1.5}{Gestionar Espacios}. 
	
	\UCpaso [\UCsist] Verifica que exista al menos un tipo de laboratorio registrado. \refTray{A}

	\UCpaso [\UCsist] Carga la información previamente registrada de la Unidad de Aprendizaje.

	\UCpaso[\UCsist] Solicita la información a través de la pantalla \cdtIdRef{IUI1.5.1}{Modificar Espacio}.
	
	\UCpaso [\UCactor] Completa la información solicitada. \label{CUI1.5.1:CompletaInfo}
	
	\UCpaso [\UCactor] Solicita modificar la información oprimiendo el botón \cdtButton{Aceptar}. \refTray{B}
	
	\UCpaso [\UCsist] Verifica que se hayan completado los campos obligatorios con base en la regla de negocios \cdtIdRef{RN-S1}{Datos requeridos}. \refTray{C}
		
	\UCpaso [\UCsist] Verifica que no haya otro Espacio registrado con el mismo nombre, con base en la regla de negocios \cdtIdRef{RN-S2}{Unicidad de elementos}. \refTray{D}
	
	\UCpaso [\UCsist] Modifica la información del espacio.
	
	\UCpaso [\UCsist] Muestra el mensaje \cdtIdRef{MSG1}{Operación Exitosa} en la pantalla \cdtIdRef{IUI1.5}{Gestionar Espacios}, indicando que la información del espacio se modificó correctamente.	
	
\end{UCtrayectoria}

\begin{UCtrayectoriaA}[Fin del caso de uso]{A}{El catálogo tipo de laboratorio no tiene información.}
	\UCpaso [\UCsist] Muestra el mensaje \cdtIdRef{MSG7}{No existe información necesaria en el sistema} en la pantalla \cdtIdRef{IUI1.5}{Gestionar Espacios}. 
\end{UCtrayectoriaA}

\begin{UCtrayectoriaA}[Fin del caso de uso]{B}{El actor desea cancelar la operación.}
	\UCpaso [\UCsist] Muestra la pantalla \cdtIdRef{IUI1.5}{Gestionar Espacios}. 
\end{UCtrayectoriaA}

\begin{UCtrayectoriaA}{C}{No se completaron los campos obligatorios.}
	\UCpaso [\UCsist] Muestra el mensaje \cdtIdRef{MSG3}{Falta dato obligatorio} en la pantalla \cdtIdRef{IUI1.5.2}{Modificar Espacio}.
	
	\UCpaso Regresa al paso \ref{CUA1.1:CompletaInfo} de la trayectoria principal.
\end{UCtrayectoriaA}

\begin{UCtrayectoriaA}{D}{El nombre del espacio ya fue registrado.}
	\UCpaso [\UCsist] Muestra el mensaje \cdtIdRef{MSG6}{Elemento duplicado} en la pantalla \cdtIdRef{IUI1.5.2}{Modificar Espacio}.
	
	\UCpaso Regresa al paso \ref{CUA1.1:CompletaInfo} de la trayectoria principal.
\end{UCtrayectoriaA}