
\begin{UseCase}{CUEE1.3}{Eliminar estructura educativa}{	
		
	Permite al actor eliminar una estructura educativa que se registró previamente, y que se hizo por error o ya no se utilizará.
}
\UCccitem{Versión}{0.1}
\UCccsection{Administración de Requerimientos}
\UCccitem{Autor}{Brenda Gómez Caballero}
\UCccitem{Evaluador}{}
\UCccitem{Operación}{Eliminación}
\UCccitem{Prioridad}{Media}
\UCccitem{Complejidad}{Baja}
\UCccitem{Volatilidad}{Muy baja}
\UCccitem{Madurez}{Alta}
\UCccitem{Estatus}{Edición}
\UCccitem{Fecha del último estatus}{}

% Copie y pegue este bloque tantas veces como revisiones tenga el caso de uso.
% Esta sección la debe llenar solo el Revisor
%--------------------------------------------------------
\UCccsection{Revisión Versión 0.1} % Anote la versión que se revisó.
\UCccitem{Fecha}{}
\UCccitem{Evaluador}{}
\UCccitem{Resultado}{}
\UCccitem{Observaciones}{
	\begin{Titemize}
		\Titem \DONE 
		\Titem \DONE 
		\Titem \DONE 
	\end{Titemize}   
}
%-------------------------------------------------------------------
\UCsection{Atributos}
	\UCitem{Actor(es)}{
		\begin{UClist}
			\UCli \cdtRef{Actor:SA}{Subdirector Académico}
		\end{UClist} 
	}
	\UCitem{Propósito}{Eliminar una estructura educativa que fue registrada por error o que no se utilizará más. Así, la información se mantendrá actualizada.}
	\UCitem{Entradas}{
		\begin{UClist}
			\UCli Ninguna
		\end{UClist}
	}
	\UCitem{Salidas}{
		\begin{UClist}
			\UCli \cdtIdRef{MSG1}{Operación Exitosa}: Se muestra en la pantalla \cdtIdRef{IUEE1}{Gestionar Estructura Educativa} indicando que la estructura educativa se eliminó correctamente.
		\end{UClist}
	}
	\UCitem{Precondiciones}{
		\begin{UClist}
			\UCli Que a la estructura educativa no se le haya registrado algún elemento.
		\end{UClist}
	}	
	\UCitem{Postcondiciones}{
		\begin{UClist}
			\UCli Ninguna	
		\end{UClist}
	}
	\UCitem{Reglas de negocio}{
		\begin{UClist}
			\UCli Ninguna
		\end{UClist}
	}
	\UCitem{Errores}{
		\begin{UClist}
			\UCli Ninguno
		\end{UClist}
	}
	\UCitem{Tipo}{Secundario, extiende del caso de uso \cdtIdRef{CUEE1}{Gestionar estructura educativa}.}
\end{UseCase}

\begin{UCtrayectoria}
	\UCpaso [\UCactor] Solicita eliminar una estructura educativa oprimiendo el botón \cdtButton{Eliminar} de la pantalla \cdtIdRef{IUEE1}{Gestionar Estructura Educativa}.
	
	\UCpaso [\UCsist] Verifica que la estructura educativa no haya sido asociado con uno o más elementos. \refTray{A}
	
	\UCpaso [\UCsist] Solicita la confirmación a través del mensaje \cdtIdRef{MSG2}{Eliminar Elemento}.
	
	\UCpaso [\UCactor] Confirma eliminar la estructura educativa oprimiendo el botón \cdtButton{Si}.
	
	\UCpaso [\UCsist] Elimina la estructura educativa.
	
	\UCpaso [\UCsist] Muestra el mensaje \cdtIdRef{MSG1}{Operación Exitosa} en la pantalla \cdtIdRef{IUEE1}{Gestionar Estructura Educativa} indicando que la estructura educativa se eliminó correctamente.	
	
\end{UCtrayectoria}

\begin{UCtrayectoriaA}[Fin del caso de uso]{A}{La estructura educativa ya fue asociada con uno o más elementos.}
	\UCpaso [\UCsist] Muestra el mensaje \cdtIdRef{MSG5}{No es posible eliminar un elemento} en la pantalla \cdtIdRef{IUEE1}{Gestionar Estructura Educativa}. 
\end{UCtrayectoriaA}
