\newpage

\section{Modelo de comportamiento del módulo: Gestionar Grupos \label{chp:gestionarGrupos}}
En este capítulo se describen los casos de uso referentes al registro, modificación y eliminación de la información de los grupos en los que se imparten las clases en la ESCOM. \bigskip

     \begin{objetivos}[Elementos de un caso de uso]
	\item {\bf Resumen:} Descripción textual del caso de uso.
	\item {\bf Actores:} Lista de los 
	 que intervienen en el caso de uso.
	\item {\bf Propósito:} Una breve descripción del objetivo que busca el actor al ejecutar el caso de uso.
	\item {\bf Entradas:} Lista de los datos de entrada requeridos durante la ejecución del caso de uso.
	\item {\bf Salidas:} Lista de los datos de salida que presenta el sistema durante la ejecución del caso de uso.
	\item {\bf Precondiciones:} Descripción de las operaciones o condiciones que se deben cumplir previamente para que el caso de uso pueda ejecutarse correctamente.
	\item {\bf Postcondiciones:} Lista de los cambios que ocurrirán en el sistema después de la ejecución del caso de uso y de las consecuencias en el sistema.
	\item {\bf Reglas de negocio:} Lista de las reglas que describen, limitan o controlan algún aspecto del negocio del caso de uso.
	\item {\bf Errores:} Lista de los posibles errores que pueden surgir durante la ejecución del caso de uso.
	\item {\bf Trayectorias:} Secuencia de los pasos que ejecutará el caso de uso.
    \end{objetivos}

	\cfinput{ModeloComportamiento/EstructuraEducativa/cuee1_1/cu}

	\cfinput{ModeloComportamiento/EstructuraEducativa/cuee1_2/cu}
	
	\cfinput{ModeloComportamiento/EstructuraEducativa/cuee1_3/cu}	
	
	\cfinput{ModeloComportamiento/EstructuraEducativa/cuap1/cu}

	\cfinput{ModeloComportamiento/EstructuraEducativa/cuap1_1/cu}
	
	\cfinput{ModeloComportamiento/EstructuraEducativa/cuap1_2/cu}

	\cfinput{ModeloComportamiento/EstructuraEducativa/cuap1_3/cu}
	
	\cfinput{ModeloComportamiento/EstructuraEducativa/cuap1_4/cu}
	
	\cfinput{ModeloComportamiento/EstructuraEducativa/cug1/cu}
	
	\cfinput{ModeloComportamiento/EstructuraEducativa/cug1_1/cu}

	\cfinput{ModeloComportamiento/EstructuraEducativa/cug1_2/cu}
	
	\cfinput{ModeloComportamiento/EstructuraEducativa/cug1_3/cu}
%===========================================================
