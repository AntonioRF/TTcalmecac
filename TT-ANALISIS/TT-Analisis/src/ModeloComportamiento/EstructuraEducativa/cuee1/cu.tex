
\begin{UseCase}{CUEE1}{Gestionar estructura educativa}{	
	Permite al actor registrar un nuevo profesor, consultar, modificar y eliminar a los profesores existentes de la ESCOM. Esto permitirá mantener la información actualizada y poder asociarlos con unidades de aprendizaje, y así generar los horarios.
}

\UCccitem{Versión}{0.1}
\UCccsection{Administración de Requerimientos}
\UCccitem{Autor}{Brenda Gómez Caballero}
\UCccitem{Evaluador}{}
\UCccitem{Operación}{Gestión}
\UCccitem{Prioridad}{Alta}
\UCccitem{Complejidad}{Baja}
\UCccitem{Volatilidad}{Muy baja}
\UCccitem{Madurez}{Alta}
\UCccitem{Estatus}{Edición}
\UCccitem{Fecha del último estatus}{30 de Septiembre de 2018}

% Copie y pegue este bloque tantas veces como revisiones tenga el caso de uso.
% Esta sección la debe llenar solo el Revisor
%--------------------------------------------------------
\UCccsection{Revisión Versión 0.1} % Anote la versión que se revisó.
\UCccitem{Fecha}{}
\UCccitem{Evaluador}{}
\UCccitem{Resultado}{}
\UCccitem{Observaciones}{
	\begin{Titemize}
		\Titem \DONE 
		\Titem \DONE 
		\Titem \DONE 
	\end{Titemize}   
}
%-------------------------------------------------------------------
	\UCsection{Atributos}
	\UCitem{Actor(es)}{	\begin{UClist}
			\UCli \cdtRef{Actor:SA}{Subdirector Académico}
			\UCli \cdtRef{Actor:C}{Capturista}
		\end{UClist} 
	}
	\UCitem{Propósito}{Llevar a cabo las acciones disponibles con el fin de generar la estructura educativa para la ESCOM, haciendo uso de la información necesaria.}
	\UCitem{Entradas}{
		\begin{UClist}
			\UCli Ninguna
		\end{UClist}
	}
	\UCitem{Salidas}{
		\begin{UClist}
			\UCli Tabla que muestra: \cdtRef{ee:nombre}{Nombre}, \cdtRef{ee:cicloEscolar}{Ciclo escolar} de la estructura educativa.
		\end{UClist}
	}
	\UCitem{Precondiciones}{
		\begin{UClist}
			\UCli Ninguna
		\end{UClist}
	}
	\UCitem{Postcondiciones}{
		\begin{UClist}
			\UCli Ninguna	
		\end{UClist}
	}
	\UCitem{Reglas de negocio}{
		\begin{UClist}
			\UCli Ninguna
		\end{UClist}
	}
	\UCitem{Errores}{
		\begin{UClist}
			\UCli Ninguno
		\end{UClist}
	}
	\UCitem{Tipo}{Primario}
\end{UseCase}

\begin{UCtrayectoria}
	
	\UCpaso [\UCactor] Selecciona la opción ''Estructura Educativa'' del menú \textbf{Estructura Educativa}.
	
	\UCpaso [\UCsist] Obtiene el nombre y ciclo escolar de la estructura educativa.
	
	\UCpaso [\UCsist] Ordena de forma descendente los registros de estructura educativa mediante el ciclo escolar.
	
	\UCpaso[\UCsist] Muestra la pantalla \cdtIdRef{IUEE1}{Gestionar Estructura Educativa} con la información obtenida con los botones \cdtButton{Modificar}, \cdtButton{Eliminar},  \cdtButton{Asociar profesores}, \cdtButton{Asociar grupos}, \cdtButton{Horario}, \cdtButton{Consultar} y el ícono \btnRegistrar. \label{cuee1:gestionar}
\end{UCtrayectoria}

\subsection{Puntos de extensión}

\UCExtensionPoint 
{El actor requiere registrar una nueva estructura educativa.}
{Paso \ref{cuee1:gestionar} de la Trayectoria Principal}
{\cdtIdRef{CUEE1.1}{Registrar estructura educativa}}

\UCExtensionPoint 
{El actor requiere modificar una estructura educativa previamente registrada.}
{Paso \ref{cuee1:gestionar} de la Trayectoria Principal}
{\cdtIdRef{CUEE1.2}{Modificar estructura educativa}}

\UCExtensionPoint 
{El actor requiere eliminar una estructura educativa previamente registrada.}
{Paso \ref{cuee1:gestionar} de la Trayectoria Principal}
{\cdtIdRef{CUEE1.3}{Eliminar estructura educativa}}

\UCExtensionPoint 
{El actor requiere configurar el horario para la estructura educativa previamente registrada.}
{Paso \ref{cuee1:gestionar} de la Trayectoria Principal}
{\cdtIdRef{CUEE1.4}{Configurar horario}}

\UCExtensionPoint 
{El actor requiere consultar el horario para la estructura educativa previamente registrada.}
{Paso \ref{cuee1:gestionar} de la Trayectoria Principal}
{\cdtIdRef{CUEE1.5}{Consultar horario}}

