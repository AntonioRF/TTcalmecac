
\begin{UseCase}{CUEE1.4}{Configurar horario}{	
		
	Una vez que se ha realizado la asociación de profesores con las unidades de aprendizaje y se han creado los grupos, el actor puede comenzar con la configuración del horario.
	 
	Permite al actor configurar los parámetros que solicita el algoritmo, de modo que, este genere la estructura educativa usando la información asociada.
}
\UCccitem{Versión}{0.1}
\UCccsection{Administración de Requerimientos}
\UCccitem{Autor}{Brenda Gómez Caballero}
\UCccitem{Evaluador}{}
\UCccitem{Operación}{Registrar}
\UCccitem{Prioridad}{Alta}
\UCccitem{Complejidad}{Baja}
\UCccitem{Volatilidad}{Muy baja}
\UCccitem{Madurez}{Alta}
\UCccitem{Estatus}{Edición}
\UCccitem{Fecha del último estatus}{}

% Copie y pegue este bloque tantas veces como revisiones tenga el caso de uso.
% Esta sección la debe llenar solo el Revisor
%--------------------------------------------------------
\UCccsection{Revisión Versión 0.1} % Anote la versión que se revisó.
\UCccitem{Fecha}{}
\UCccitem{Evaluador}{}
\UCccitem{Resultado}{}
\UCccitem{Observaciones}{
	\begin{Titemize}
		\Titem \DONE 
		\Titem \DONE 
		\Titem \DONE 
	\end{Titemize}   
}
%-------------------------------------------------------------------
\UCsection{Atributos}
	\UCitem{Actor(es)}{
		\begin{UClist}
			\UCli \cdtRef{Actor:SA}{Subdirector Académico}
		\end{UClist} 
	}
	\UCitem{Propósito}{Definir los parámetros que permiten generar la estructura educativa haciendo uso del algoritmo.}
	\UCitem{Entradas}{
		\begin{UClist}
			\UCli Número de estructuras a generar
		\end{UClist}
	}
	\UCitem{Salidas}{
		\begin{UClist}
			\UCli \cdtIdRef{MSG1}{Operación Exitosa}: Se muestra en la pantalla \cdtIdRef{IUEE1}{Gestionar Estructura Educativa} indicando que el algoritmo se generó correctamente.
		\end{UClist}
	}
	\UCitem{Precondiciones}{
		\begin{UClist}
			\UCli Que se haya registrado al menos una asociación de un profesor con una unidad de aprendizaje.
			\UCli Que se haya registrado al menos un grupo. 
		\end{UClist}
	}	
	\UCitem{Postcondiciones}{
		\begin{UClist}
			\UCli Ninguna	
		\end{UClist}
	}
	\UCitem{Reglas de negocio}{
		\begin{UClist}
			\UCli Ninguna
		\end{UClist}
	}
	\UCitem{Errores}{
		\begin{UClist}
			\UCli Ninguno
		\end{UClist}
	}
	\UCitem{Tipo}{Secundario, extiende del caso de uso \cdtIdRef{CUEE1}{Gestionar estructura educativa}.}
\end{UseCase}

\begin{UCtrayectoria}
	\UCpaso [\UCactor] Solicita configurar el horario oprimiendo el botón \cdtButton{Horario} de la pantalla \cdtIdRef{IUEE1}{Gestionar Estructura Educativa}.
	
	\UCpaso [\UCsist] Verifica que la estructura educativa no haya sido asociado con uno o más elementos. \refTray{A}
	
	\UCpaso [\UCsist] Solicita la confirmación a través del mensaje \cdtIdRef{MSG2}{Eliminar Elemento}.
	
	\UCpaso [\UCactor] Confirma eliminar la estructura educativa oprimiendo el botón \cdtButton{Si}.
	
	\UCpaso [\UCsist] Elimina la estructura educativa.
	
	\UCpaso [\UCsist] Muestra el mensaje \cdtIdRef{MSG1}{Operación Exitosa} en la pantalla \cdtIdRef{IUEE1}{Gestionar Estructura Educativa} indicando que la estructura educativa se eliminó correctamente.	
	
\end{UCtrayectoria}

\begin{UCtrayectoriaA}[Fin del caso de uso]{A}{La estructura educativa ya fue asociada con uno o más elementos.}
	\UCpaso [\UCsist] Muestra el mensaje \cdtIdRef{MSG5}{No es posible eliminar un elemento} en la pantalla \cdtIdRef{IUEE1}{Gestionar Estructura Educativa}. 
\end{UCtrayectoriaA}
