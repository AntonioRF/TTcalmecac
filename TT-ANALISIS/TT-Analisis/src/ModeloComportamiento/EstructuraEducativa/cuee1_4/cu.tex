
\begin{UseCase}{CUEE1.4}{Configurar horario}{	
		
	Una vez que se ha realizado la asociación de profesores con las unidades de aprendizaje y se han creado los grupos, el actor puede comenzar con la configuración del horario.
	 
	Permite al actor configurar los parámetros que solicita el algoritmo, de modo que, este genere la estructura educativa usando la información asociada.
}
\UCccitem{Versión}{0.1}
\UCccsection{Administración de Requerimientos}
\UCccitem{Autor}{Brenda Gómez Caballero}
\UCccitem{Evaluador}{}
\UCccitem{Operación}{Registrar}
\UCccitem{Prioridad}{Alta}
\UCccitem{Complejidad}{Baja}
\UCccitem{Volatilidad}{Muy baja}
\UCccitem{Madurez}{Alta}
\UCccitem{Estatus}{Edición}
\UCccitem{Fecha del último estatus}{}

% Copie y pegue este bloque tantas veces como revisiones tenga el caso de uso.
% Esta sección la debe llenar solo el Revisor
%--------------------------------------------------------
\UCccsection{Revisión Versión 0.1} % Anote la versión que se revisó.
\UCccitem{Fecha}{}
\UCccitem{Evaluador}{}
\UCccitem{Resultado}{}
\UCccitem{Observaciones}{
	\begin{Titemize}
		\Titem \DONE 
		\Titem \DONE 
		\Titem \DONE 
	\end{Titemize}   
}
%-------------------------------------------------------------------
\UCsection{Atributos}
	\UCitem{Actor(es)}{
		\begin{UClist}
			\UCli \cdtRef{Actor:SA}{Subdirector Académico}
		\end{UClist} 
	}
	\UCitem{Propósito}{Definir los parámetros para el algoritmo encargado de generar la estructura educativa.}
	\UCitem{Entradas}{
		\begin{UClist}
			\UCli \cdtRef{alg:numero}{Número} de estructuras a generar: \ioEscribir.
		\end{UClist}
	}
	\UCitem{Salidas}{
		\begin{UClist}
			\UCli \cdtIdRef{MSG1}{Operación Exitosa}: Se muestra en la pantalla \cdtIdRef{IUEE1}{Gestionar Estructura Educativa} indicando que el algoritmo se generó correctamente.
		\end{UClist}
	}
	\UCitem{Precondiciones}{
		\begin{UClist}
			\UCli Que se haya registrado al menos una asociación de un profesor con una unidad de aprendizaje.
			\UCli Que se haya registrado al menos un grupo. 
		\end{UClist}
	}	
	\UCitem{Postcondiciones}{
		\begin{UClist}
			\UCli Ninguna	
		\end{UClist}
	}
	\UCitem{Reglas de negocio}{
		\begin{UClist}
			\UCli \cdtIdRef{RN-S1}{Datos requeridos}
		\end{UClist}
	}
	\UCitem{Errores}{
		\begin{UClist}
			\UCli \cdtIdRef{MSG3}{Falta dato obligatorio}: Se muestra en la pantalla \cdtIdRef{IUEE1.4}{Configurar horario} indicando que faltan campos obligatorios por completar.
		\end{UClist}
	}
	\UCitem{Tipo}{Secundario, extiende del caso de uso \cdtIdRef{CUEE1}{Gestionar estructura educativa}.}
\end{UseCase}

\begin{UCtrayectoria}
	\UCpaso [\UCactor] Solicita configurar el horario oprimiendo el botón \cdtButton{Horario} de la pantalla \cdtIdRef{IUEE1}{Gestionar Estructura Educativa}.
	
	\UCpaso [\UCactor] Completa la información solicitada. \label{CUEE1.4:CompletaInfo}

	\UCpaso [\UCactor] Solicita generar el horario oprimiendo el botón \cdtButton{Generar horario}. \refTray{A}

	\UCpaso [\UCsist] Verifica que se hayan completado los campos obligatorios con base en la regla de negocios \cdtIdRef{RN-S1}{Datos requeridos}. \refTray{B}

	\UCpaso [\UCsist] Registra la información del horario.
	
	\UCpaso [\UCsist] Envía la instrucción para genera el horario.

	\UCpaso [\UCsist] Muestra el mensaje \cdtIdRef{MSG1}{Operación Exitosa} en la pantalla \cdtIdRef{IUEE1}{Gestionar Estructura Educativa}, indicando que se generó la estructura educativa mediante el algoritmo correctamente.
	
\end{UCtrayectoria}


\begin{UCtrayectoriaA}[Fin del caso de uso]{A}{El actor desea cancelar la operación.}
	\UCpaso [\UCactor] Solicita cancelar la operación oprimiendo el botón \cdtButton{Cancelar}.
	
	\UCpaso [\UCsist] Muestra la pantalla \cdtIdRef{IUEE1}{Gestionar Estructura Educativa}. 
\end{UCtrayectoriaA}

\begin{UCtrayectoriaA}{B}{No se completaron los campos obligatorios.}
	\UCpaso [\UCsist] Muestra el mensaje \cdtIdRef{MSG3}{Falta dato obligatorio} en la pantalla \cdtIdRef{IUEE1.4}{Configurar horario}.
	
	\UCpaso Regresa al paso \ref{CUEE1.4:CompletaInfo} de la trayectoria principal.
\end{UCtrayectoriaA}
