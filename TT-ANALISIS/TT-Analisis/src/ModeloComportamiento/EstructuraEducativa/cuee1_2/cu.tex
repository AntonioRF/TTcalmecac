
\begin{UseCase}{CUEE1.2}{Modificar estructura educativa}{	
		
	La estructura educativa es el nombre que recibe el proceso para realizar los horarios en los que se impartirán las clases en la ESCOM.
		
	Permite al actor modificar el registro de estructura educativa previamente registrada. De modo que, estas se mantendrán actualizadas y así se identificarán las diferentes opciones de horario que pueden generarse. 
}
\UCccitem{Versión}{0.1}
\UCccsection{Administración de Requerimientos}
\UCccitem{Autor}{Brenda Gómez Caballero}
\UCccitem{Evaluador}{}
\UCccitem{Operación}{Edición}
\UCccitem{Prioridad}{Alta}
\UCccitem{Complejidad}{Baja}
\UCccitem{Volatilidad}{Muy baja}
\UCccitem{Madurez}{Alta}
\UCccitem{Estatus}{Edición}
\UCccitem{Fecha del último estatus}{}

% Copie y pegue este bloque tantas veces como revisiones tenga el caso de uso.
% Esta sección la debe llenar solo el Revisor
%--------------------------------------------------------
\UCccsection{Revisión Versión 0.1} % Anote la versión que se revisó.
\UCccitem{Fecha}{}
\UCccitem{Evaluador}{}
\UCccitem{Resultado}{}
\UCccitem{Observaciones}{
	\begin{Titemize}
		\Titem \DONE 
		\Titem \DONE 
		\Titem \DONE 
	\end{Titemize}   
}
%-------------------------------------------------------------------
	\UCsection{Atributos}
	\UCitem{Actor(es)}{ \begin{UClist}
			\UCli \cdtRef{Actor:SA}{Subdirector Académico}
		\end{UClist} 
	}
	\UCitem{Propósito}{Actualizar la información de una estructura educativa para crear la información necesaria para generar los horarios de clases de la ESCOM.}
	\UCitem{Entradas}{
		\begin{UClist}
			\UCli \cdtRef{estructuraEducativa:nombre}{Nombre}: \ioEscribir.
			
			\UCli \cdtRef{estructuraEducativa:ciclo}{Ciclo escolar}: \ioSeleccionar.
		\end{UClist}
	}
	\UCitem{Salidas}{
		\begin{UClist}
			\UCli \cdtRef{estructuraEducativa:nombre}{Nombre}: \ioObtener.
			
			\UCli \cdtRef{estructuraEducativa:ciclo}{Ciclo escolar}: \ioObtener.
			
			\UCli \cdtIdRef{MSG1}{Operación Exitosa}: Se muestra en la pantalla \cdtIdRef{IUEE1}{Gestionar Estructura Educativa} indicando que la edición se realizó correctamente.
		\end{UClist}
	}
	\UCitem{Precondiciones}{
		\begin{UClist}
			\UCli Ninguna
		\end{UClist}
	}
	\UCitem{Postcondiciones}{
		\begin{UClist}
			\UCli Ninguna	
		\end{UClist}
	}
	\UCitem{Reglas de negocio}{
		\begin{UClist}
			\UCli \cdtIdRef{RN-S1}{Datos requeridos}
			
			\UCli \cdtIdRef{RN-S2}{Unicidad de elementos} 
		\end{UClist}
	}
	\UCitem{Errores}{
		\begin{UClist}	
			\UCli \cdtIdRef{MSG3}{Falta dato obligatorio}: Se muestra en la pantalla \cdtIdRef{IUEE1.2}{Modificar Estructura Educativa} indicando que faltan campos obligatorios por completar.
			
			\UCli \cdtIdRef{MSG6}{Elemento duplicado}: Se muestra en la pantalla \cdtIdRef{IUEE1.2}{Modificar Estructura Educativa} indicando que el ciclo ya se utilizó.
		\end{UClist}
	}
	\UCitem{Tipo}{Secundario, extiende del caso de uso \cdtIdRef{CUEE1}{Gestionar estructura educativa}.}
\end{UseCase}

\begin{UCtrayectoria}
	\UCpaso [\UCactor] Solicita modificar una estructura educativa oprimiendo el botón \cdtButton{Modificar} de la pantalla \cdtIdRef{IUEE1}{Gestionar Estructura Educativa}. 
	
	\UCpaso [\UCsist] Solicita la información a través de la pantalla \cdtIdRef{IUEE1.2}{Modificar Estructura Educativa}.
	
	\UCpaso [\UCactor] Completa la información solicitada. \label{CUEE1.2:CompletaInfo}
	
	\UCpaso [\UCactor] Solicita actualizar la información oprimiendo el botón \cdtButton{Aceptar}. \refTray{A}
	
	\UCpaso [\UCsist] Verifica que se hayan completado los campos obligatorios con base en la regla de negocios \cdtIdRef{RN-S1}{Datos requeridos}. \refTray{B}
	
	\UCpaso [\UCsist] Verifica que no se haya registrado una estructura educativa con el ciclo escolar en cuestión, con base en la regla de negocios \cdtIdRef{RN-S2}{Unicidad de elementos}. \refTray{C}
	
	\UCpaso [\UCsist] Actualiza la información de la estructura educativa.
	
	\UCpaso [\UCsist] Muestra el mensaje \cdtIdRef{MSG1}{Operación Exitosa} en la pantalla \cdtIdRef{IUEE1}{Gestionar Estructura Educativa}, indicando que la estructura educativa se actualizó correctamente.	
	
\end{UCtrayectoria}


\begin{UCtrayectoriaA}[Fin del caso de uso]{A}{El actor desea cancelar la operación.}
	\UCpaso [\UCactor] Solicita cancelar la operación oprimiendo el botón \cdtButton{Cancelar}.
	
	\UCpaso [\UCsist] Muestra la pantalla \cdtIdRef{IUEE1}{Gestionar Estructura Educativa}. 
\end{UCtrayectoriaA}

\begin{UCtrayectoriaA}{B}{No se completaron los campos obligatorios.}
	\UCpaso [\UCsist] Muestra el mensaje \cdtIdRef{MSG3}{Falta dato obligatorio} en la pantalla \cdtIdRef{IUEE1.2}{Modificar Estructura Educativa}.
	
	\UCpaso Regresa al paso \ref{CUEE1.2:CompletaInfo} de la trayectoria principal.
\end{UCtrayectoriaA}

\begin{UCtrayectoriaA}{C}{El ciclo escolar ya fue registrado en otro registro.}
	\UCpaso [\UCsist] Muestra el mensaje \cdtIdRef{MSG6}{Elemento duplicado} en la pantalla \cdtIdRef{IUEE1.2}{Modificar Estructura Educativa}.
	
	\UCpaso Regresa al paso \ref{CUEE1.2:CompletaInfo} de la trayectoria principal.
\end{UCtrayectoriaA}
