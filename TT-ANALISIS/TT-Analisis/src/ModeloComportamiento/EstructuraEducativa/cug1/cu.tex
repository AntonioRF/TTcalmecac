
\begin{UseCase}{CUG1}{Gestionar grupos}{	
	Permite al actor registrar un nuevo grupo, modificar y eliminar a los grupos existentes de la ESCOM. Esto permitirá mantener la información actualizada y poder asociarlos con espacios, y así generar los horarios.
}

\UCccitem{Versión}{0.1}
\UCccsection{Administración de Requerimientos}
\UCccitem{Autor}{Brenda Gómez Caballero}
\UCccitem{Evaluador}{}
\UCccitem{Operación}{Gestión}
\UCccitem{Prioridad}{Alta}
\UCccitem{Complejidad}{Baja}
\UCccitem{Volatilidad}{Muy baja}
\UCccitem{Madurez}{Alta}
\UCccitem{Estatus}{Edición}
\UCccitem{Fecha del último estatus}{}

% Copie y pegue este bloque tantas veces como revisiones tenga el caso de uso.
% Esta sección la debe llenar solo el Revisor
%--------------------------------------------------------
\UCccsection{Revisión Versión 0.1} % Anote la versión que se revisó.
\UCccitem{Fecha}{}
\UCccitem{Evaluador}{}
\UCccitem{Resultado}{}
\UCccitem{Observaciones}{
	\begin{Titemize}
		\Titem \DONE 
		\Titem \DONE 
		\Titem \DONE 
	\end{Titemize}   
}
%-------------------------------------------------------------------
	\UCsection{Atributos}
	\UCitem{Actor(es)}{	\begin{UClist}
			\UCli \cdtRef{Actor:SA}{Subdirector Académico}
		\end{UClist} 
	}
	\UCitem{Propósito}{}
	\UCitem{Entradas}{
		\begin{UClist}
			\UCli Ninguna
		\end{UClist}
	}
	\UCitem{Salidas}{
		\begin{UClist}
			\UCli Tabla que muestra: \cdtRef{grupo:nombre}{Nombre}, \cdtRef{espacio:clave}{Salón} de ls grupos.
		\end{UClist}
	}
	\UCitem{Precondiciones}{
		\begin{UClist}
			\UCli Ninguna
		\end{UClist}
	}
	\UCitem{Postcondiciones}{
		\begin{UClist}
			\UCli Ninguna	
		\end{UClist}
	}
	\UCitem{Reglas de negocio}{
		\begin{UClist}
			\UCli Ninguna
		\end{UClist}
	}
	\UCitem{Errores}{
		\begin{UClist}
			\UCli Ninguno
		\end{UClist}
	}
	\UCitem{Tipo}{Primario}
\end{UseCase}

\begin{UCtrayectoria}
	
	\UCpaso [\UCactor] Solicita gestionar los grupos oprimiendo el botón \cdtButton{Asociar grupos} de la pantalla \cdtIdRef{IUEE1}{Gestionar Estructura Educativa}.
	
	\UCpaso [\UCsist] Obtiene el nombre y salón de los grupos.
	
	\UCpaso [\UCsist] Ordena de forma ascendente los registros mediante el nombre del grupo.
	
	\UCpaso[\UCsist] Muestra la pantalla \cdtIdRef{IUG1}{Gestionar Grupos} con la información obtenida con los botones \cdtButton{Modificar} y \cdtButton{Eliminar}, con el ícono \btnRegistrar. \label{cug1:gestionar}
\end{UCtrayectoria}


\subsection{Puntos de extensión}

\UCExtensionPoint 
{El actor requiere registrar un nuevo grupo.}
{Paso \ref{cug1:gestionar} de la Trayectoria Principal}
{\cdtIdRef{CUG1.1}{Registrar grupo}}

\UCExtensionPoint 
{El actor requiere modificar un grupo previamente registrado.}
{Paso \ref{cug1:gestionar} de la Trayectoria Principal}
{\cdtIdRef{CUG1.2}{Modificar grupo}}

\UCExtensionPoint 
{El actor requiere eliminar un grupo previamente registrado.}
{Paso \ref{cug1:gestionar} de la Trayectoria Principal}
{\cdtIdRef{CUG1.3}{Eliminar grupo}}


